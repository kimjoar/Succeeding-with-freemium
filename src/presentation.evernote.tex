\section{Case 2 --- Evernote}

Evernote\footnote{The presentation of Evernote is based on their sharing of information in \citet{libin2010,libin2010b,libin2010c,libin2010d,libin2010e,grove2010,darlin2009,sattersten2010}.} is a suite of software and services designed for notetaking and archiving, based on the \abbr{SaaS} model. Phil Libin, \abbr{CEO} of Evernote, describes it as an \q{external brain} and a \q{universal memory drawer.} The company launched in June, 2008, and has more than 5 million users. Of these users, about 160,000, and thus 3.2\%, are paying users. Evernote has received \$20 million \abbr{USD} in funding from \emph{Sequoia Capital}, one of the most prominent venture capital firms in the world.

The Evernote service has three price plans: free, \$5 a month and \$45 a year. They are, however, very committed to their free option, stating that \q{We want most people using Evernote for free.} Their premium offering is one of Evernote's five revenue models but stands for almost 90\% on revenue.

Evernote does no advertising at all, as \q{our product is our marketing.} This is interesting as their software is \q{not a social app,} as Libin himself states it, are as there are therefore no network externalities. However, according to Libin \q{98\% of our growth comes from people telling their friends}. Thus, Evernote has significant word-of-mouth effects.

For their customers, Evernote have found that the longer they use the service, the more likely they are to upgrade to the paid version. Thus, the longer they remain customers, the more profitable they become. Usually Evernote have seen about 0.5\% of new users becoming paying customers in the first month. In the same first month they usually lose 50-60\% of the new users, and usually an additional 5-10\% the second month. However, from that point on they almost don't lose any customers at all. Looking 29 months in on a group of users that joined the service the same month, more than 20\% have converted to paying. According to Libin, \q{the more memories users store in Evernote, the more invested they become.} In the same vein he has said for freemium to work a company \q{need a product that increases in value over time.}

Based on this gradually increasing conversion rate and as the users of their paid version is \q{growing much faster than our free user base}, Evernote is experiencing an exponential growth in revenue --- more specifically they have experienced month-over-month revenue growths of almost double the month-over-month user growth.

The company has a \$0.25 \abbr{USD} revenue and variable expenses of \$0.09 \abbr{USD} per month per active user, and thus a gross margin of \$0.16 \abbr{USD}. Within these numbers are many interesting points. First of all, as they have a low revenue per active user per month, they need to acquire customers very cheaply. The company in fact does no advertising at all, only relying on organic growth, and primarily word-of-mouth. Secondly, they have low variable expenses. According to Libin, they modeled Evernote to be profitable at a 1\% conversion rate. Thus, the company is built to have a very low variable expense per user. 

Connecting these costs and the fact that Evernote's conversion rate increases with how long they have used the service, Evernote focuses on having a high long-term retention rate. Libin explains this as \q{[the customers] stay a long time, they convert more as they age, and their costs stay flat --- that's what you need to make freemium work.}

Evernote has, according to themselves, built their premium version specifically for \term{power users}, \ie those that very actively use the service, and included several features that are not necessary for most users. An interesting point is that they see an increase in revenue when adding additional premium features, no matter if the converting customer use them or not. Interestingly, this means that conversion from free to premium is not based on use. A possible explanation for this, is that it can be based on the perceived value of the product increasing with the functionality upgrade.

One thing the company states over and over again publicly, is that \q{People pay for what they love.} They are very focused on value creation. An interesting point with regards to this, is that Libin has said that \q{we just haven't had time to get into heavy price theory yet,} and that the primary reason why people upgrade to the premium version is because \q{they love Evernote.} Explaining their pricing, he says that \q{We decided to go with a single, low price for the premium version to keep the decision for users as simple as possible.}

Evernote has not publicly stated their net income, only the fact that they have a positive gross margin, and that it is increasing.

% http://blog.evernote.com/2010/10/19/evernote-raises-20-million-led-by-sequoia-capital/

% The free service is supported by advertising, on both the web interface and in the application. The premium service allows the user to turn off this advertising.
% 
% \url{http://blogs.wsj.com/venturecapital/2009/10/25/sequoia-capital-charges-hard-through-recession/}
% 
% \url{http://www.nytimes.com/2009/08/30/business/30ping.html}:
% 
% \q{Free is not a loss leader, if we can get a small percentage of users to pay we start to make money.}
% 
% \url{http://www.fastcompany.com/magazine/147/next-tech-remember-the-money.html}:
% 
% \url{http://gigaom.com/2010/03/26/case-studies-in-freemium-pandora-dropbox-evernote-automattic-and-mailchimp/}:
% 
% http://www.youtube.com/watch?v=NvQeb-MGXCY
% 
% Fra \url{http://www.youtube.com/watch?v=5dzLIqoq4nM}:
% 
% \begin{items}
%   \item \q{We don't think of free as a loss leader.}
%   \item \q{Multi-platform use is the greatest predictor for user conversion to premium.}
%   \item Bruker eksempel fra mars 2008. Halvparten haller av iløpet av første måned, 5-10\% neste måned, og så er det helt flatt. Retention: 11 000 people.
%   \item \q{Even though the number of active users declines as the cohort ages, the total revenue keeps increasing.}
%   \item \q{We have not focused on the conversion rates}
%   \item First month: 2-3 cent per aktive bruker. 2 år: 70 cents. 
%   \item Low variable cost
%   \item It is a numbers game
% \end{items}
% 
% \url{http://eu.techcrunch.com/2010/12/10/evernote-ceo-phil-libin-at-le-web-tctv/}:
% 
% \url{http://www.mobile-ent.biz/news/39807/LeWeb-10-Evernote-has-160000-paying-users}:
% 
% Playfish co-founder Sebastien de Halleux said at LeWeb this week that Facebook social games tend to get between 1\% and 5\% of players paying for content.
% 
% \url{http://www.youtube.com/watch?v=Oj-ougERrUQ}:
% 
% \begin{items}
%   \item \q{People pay for what they love.}
%   \item The percentage doesn't matter, but the total number that pays matters.
% \end{items}
% 
% \url{http://mashable.com/2010/12/10/evernote-growth/}:
% 
% We've found that the more content we provide, the more engaged our users become, and the more likely they are to tell a friend and convert to Evernote Premium.
% 
% \url{http://toddsattersten.com/2010/02/fixed-to-flexible-interview-with-evernote-ceo-phil-libin.html}:
% 
% The gross margin comes out to about 53\%. The gross margin increases every month because revenue per user grows (conversion rates go up because long-time users are more likely to convert) while variable expenses per user decline (Moore's law + efficiencies of scale). 
% 
% We launched the service into closed beta in February of 2008. Gross margin went positive in January 2009.
% 
% 35\% active users (defined as being active a given month)
% 2,335,676 total users
% 41,598 premium users
% 38972 users per server
% 4 sysadmins
% \$68,641 - Total variable expenses (hardware + software + hosting + network + operations staff + support staff)
% \$145,000 - Total revenue from active premium users
% 
% From this we can deduce some interesting numbers:
% * free/paying ratio (all users) - 1,7\%
% * free/paying ratio (active users) - 2.2\%
% * revenue per user (all users) - \$0,06
% * revenue per active premium user - \$3,5
% * Variable expense per user - \$0,29
% * 38927 users per server
% * 583919 users per sysadmin
% 
% I particularly find the \$0,06 revenue per user interesting, since it gives a rough number to work with if you're doing a freemium website. Eg. if you need \$10.000 in monthly revenue to break even you will need 166.666 users.
% 
% \q{Invest heavily in the product; focusing on things that will make your customers love you and things that will keep your variable costs low when you scale. Make your product free so that you don't have to pay for traditional marketing.}
% 
% \q{Slowly introduce paid features but always keep the \q{main} product free. Get to positive gross margins}
% 
% The real medium-term cost savings comes from efficiencies of scale. For example, we currently have about 60 servers in the data center and four operations guys to maintain them (both included in the variable costs for gross margin calculations). When we get to 600 servers, we won't need 30 ops guys, probably 10 will be enough.
% 
% Here are three things I think about [when rolling out a freemium model]:
% 
% 1. Make a product that a billion people will fall in love with and use for the rest of their lives.
% 2. Make it easy for a single-digit percentage of them to pay you a few bucks a month once in a while.
% 3. Make sure your variable costs are low enough that you can make a mountain of profit if you get \#1 and \#2 right.
% 
% If you can't see how you'll do all three things, go with another business model.
% 
% \url{http://www.bbc.co.uk/blogs/technology/2009/12/freemium_can_spotify_learn_fro.html}:
% 
% \q{We've decided to experiment with full transparency on the numbers}
% 
% the few adverts that come with the free service only produce negligible revenue
% 
% \url{http://vimeo.com/11932184}:
% 
% Great long term retention rate, product that increases in value over time, low variable costs.