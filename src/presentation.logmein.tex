\section{Case 1 --- LogMeIn}

LogMeIn\footnote{The presentation of LogMeIn is based on their sharing of information in \citet{logmein2009,logmein2010,moore2010,seitz2010,vance2009}.} delivers a suite of software services that provides remote access to computers over the Internet for both end users and professional help desk personnel. They launched in 2003 and are now publicly traded at NASDAQ, with a market capitalization of more than \$1 billion \abbr{USD}.

LogMeIn has a suite of nine applications, all of which are \term{Software-as-a-Service} (\abbr{SaaS}).\footnote{Software-as-a-Service is defined by \citet[\p{185}]{saaksjarvi2005} as \q{time and location independent online access to a remotely managed server application, that permits concurrent utilization of the same application installation by a large number of independent users.}} Two of these nine applications, \emph{LogMeIn Free} and \emph{LogMeIn Hamachi$^2$}, have elements of free. Thus, LogMeIn uses the alternative product strategy, as discussed in \sectionref{context:freepaid}. 

At the end of Q3 2010, LogMeIn had 10.4 million users, of which 490 000 --- just above 4.7\% --- where paying customers. However, in this last quarter 12\% of their new users, 85 000 of 700 000, chose on of the premium products, almost three times the total rate. For 2009 the company reports a gross margin of 90\% --- which include the cost of the free users. This number is up from 85\% a year earlier, indicating that they increase their revenue faster than the cost of revenue grows. They also reported a net income of 12\% of their \$74.4 million \abbr{USD} revenue in 2009, which is a revenue growth of 44\% year-over-year. In addition to this, the company reports a 80 percent year-over-year retention rate among subscribers.

The company states that they \q{acquire new customers through word-of-mouth referrals from [the] existing user community and from paid, online advertising designed to attract visitors to [the] website.} Each new paying customers comes at a cost of \$300, but has an average lifetime revenue of \$1500. According to the company they can get customers at such a low cost because of their \q{efficient customer acquisition model,} in which their freemium offering is front and center --- as they state: \q{We grow our community [\ldots] by offering our popular free services,} and it is this community that \q{generate word-of-mouth referrals} and thus increase awareness for their premium services. LogMeIn views this customer acquisition cost as one of their \q{key competitive factors.}

LogMeIn is built on their proprietary and patented platform technology, which according to them gives the company a cost advantage that allows them to offer some of their services for free, even to the majority of users. They believe this lets them serve a broader user community than their competitors --- \q{we reach significantly more users which allows us to attract paying customers efficiently.} They have explained their free offering as putting a \q{neutron bomb on the competitive landscape.}

LogMeIn sell their software to both consumers (\abbr{B2C}) and businesses (\abbr{B2B}), however, both of their free products are aimed at non-commercial use, and thus primarily at consumers. All LogMeIn's premium services are subscription based, and \q{the majority of our customers subscribe to our services on an annual basis.} The company do not specify any numbers of unpaid and paid subscriptions on each service, only the sum for all of theme. However, they specify that their revenue comes primarily from SMBs, IT service providers and consumers.

During 2009 they used 48\% of their revenue on sales and marketing, thus they commit significant resources to reaching new customers, but when they compare themselves to their largest competitors they state that, while these \q{attract new customers through traditional marketing and sales efforts, [we] have focused first on building a large-scale community of users.} The primary rationale for doing this, according to the company, is that they are competing in a competitive market with low barriers to entry. They state that: \q{We believe our large user base also gives us an advantage over smaller competitors and potential new entrants into the market by making it more expensive for them to gain general market awareness.} Thus, LogMeIn competes with companies using freemium by using freemium themselves.

According to LogMeIn \abbr{CEO} Michael Simon, their free service is \q{the oxygen supply to our business.}

% \todo{Lag tabell fra customers and users-sliden i \url{http://files.shareholder.com/downloads/ABEA-358CAD/1089163783x0x427024/6ddd9123-e051-4e2d-b199-a5ebce29dea0/LogMeIn_Investor_Presentation_Q4_10.pdf}}
% 
% Unpaid sources of demand generation: \q{Direct Advertising Into Our User Community.	We have a large existing community of free users and paying customers. Users of most of our services, including our most popular service, LogMeIn Free, come to our website each time they initiate a new remote access session. We use this opportunity to promote additional premium services to them.}
% 
% Marketing initiative: \q{Social Media Marketing. We participate in online communities such as Twitter, Facebook and YouTube for the purpose of marketing, public relations and customer service. Through these online collaboration sites, we actively engage our users, learn about their wants, and foster word-of-mouth by creating and responding to content about LogMeIn events, promotions, product news and user questions.}
% 
% Marketing initiative: \q{Web-Based Seminars.	We offer free online seminars to current and prospective customers designed to educate them about the benefits of remote access, support and administration, particularly with LogMeIn, and guide them in the use of our services. We often highlight customer success stories and focus the seminar on business problems and key market and IT trends.}
% 
% \q{We believe that our large-scale user base, efficient customer acquisition model and low service delivery costs enable us to compete effectively.}
% 
% \q{LogMeIn Hamachi is offered both as a free service for non-commercial use and as a paid service for commercial use.}
% 
% Bør gå inn og se på tallene og gjøre en analyse av hvor dyre gratisbrukerne er.
% 
% \url{http://www.masshightech.com/stories/2010/06/21/daily2-LogMeIn-sees-stock-and-app-take-off-with-iPad-success.html}
% 
% \url{http://www.investors.com/NewsAndAnalysis/Article/536360/201006041730/Switching-From-Free-To-Fee-.aspx}
% 
% \url{http://bits.blogs.nytimes.com/2009/09/28/logmein-links-to-pcs-and-ford-150s/#more-20149}