\chapter{Presentation of the four cases}
\label{chapter:presentation}

Using the process outlined in the methodology, four cases was chosen --- three successful and one unsuccessful --- from an initial list of 49. This chapter presents the empirical findings for each of these cases, starting with a summary in \tableref{presentation:summary} of the rationale for choosing each case.

\begin{margintable}[h]
  \begin{tabularx}{\textwidth}{ l X }
    \toprule
    Name       & Rationale for selection \\
    \midrule
    LogMeIn    & Publicly traded with a market capitalization of \$1 billion \abbr{USD}, thus a highly successful freemium-based company. The only among the four using an alternative product strategy. \\
    \midrule
    Evernote   & Have been very public about their numbers and has an exponential revenue growth. Have no network effects, but at the same time don't spent money on advertising. Very strong word-of-mouth effects. \\
    \midrule
    MailChimp  & Had been a premium-only company for eighth years when they changed to freemium, and experienced a tremendous growth both in users and profit. \\
    \midrule
    Helpstream & One of few companies that have been very public about their failure with freemium. More business-to-business oriented than the other case companies. \\
    \bottomrule
  \end{tabularx}
  \caption{The four cases, including a rationale for why each was chosen.}
  \label{table:presentation:summary}
\end{margintable}

\section{Case 1 --- LogMeIn}

LogMeIn\footnote{The presentation of LogMeIn is based on their sharing of information in \citet{logmein2009,logmein2010,moore2010,seitz2010,vance2009}.} delivers a suite of software services that provides remote access to computers over the Internet for both end users and professional help desk personnel. They launched in 2003 and are now publicly traded at NASDAQ, with a market capitalization of more than \$1 billion \abbr{USD}.

LogMeIn has a suite of nine applications, all of which are \term{Software-as-a-Service} (\abbr{SaaS}).\footnote{Software-as-a-Service is defined by \citet[\p{185}]{saaksjarvi2005} as \q{time and location independent online access to a remotely managed server application, that permits concurrent utilization of the same application installation by a large number of independent users.}} Two of these nine applications, \emph{LogMeIn Free} and \emph{LogMeIn Hamachi$^2$}, have elements of free. Thus, LogMeIn uses the alternative product strategy, as discussed in \sectionref{context:freepaid}. 

At the end of Q3 2010, LogMeIn had 10.4 million users, of which 490 000 --- just above 4.7\% --- where paying customers. However, in this last quarter 12\% of their new users, 85 000 of 700 000, chose on of the premium products, almost three times the total rate. For 2009 the company reports a gross margin of 90\% --- which include the cost of the free users. This number is up from 85\% a year earlier, indicating that they increase their revenue faster than the cost of revenue grows. They also reported a net income of 12\% of their \$74.4 million \abbr{USD} revenue in 2009, which is a revenue growth of 44\% year-over-year. In addition to this, the company reports a 80 percent year-over-year retention rate among subscribers.

The company states that they \q{acquire new customers through word-of-mouth referrals from [the] existing user community and from paid, online advertising designed to attract visitors to [the] website.} Each new paying customers comes at a cost of \$300, but has an average lifetime revenue of \$1500. According to the company they can get customers at such a low cost because of their \q{efficient customer acquisition model,} in which their freemium offering is front and center --- as they state: \q{We grow our community [\ldots] by offering our popular free services,} and it is this community that \q{generate word-of-mouth referrals} and thus increase awareness for their premium services. LogMeIn views this customer acquisition cost as one of their \q{key competitive factors.}

LogMeIn is built on their proprietary and patented platform technology, which according to them gives the company a cost advantage that allows them to offer some of their services for free, even to the majority of users. They believe this lets them serve a broader user community than their competitors --- \q{we reach significantly more users which allows us to attract paying customers efficiently.} They have explained their free offering as putting a \q{neutron bomb on the competitive landscape.}

LogMeIn sell their software to both consumers (\abbr{B2C}) and businesses (\abbr{B2B}), however, both of their free products are aimed at non-commercial use, and thus primarily at consumers. All LogMeIn's premium services are subscription based, and \q{the majority of our customers subscribe to our services on an annual basis.} The company do not specify any numbers of unpaid and paid subscriptions on each service, only the sum for all of theme. However, they specify that their revenue comes primarily from SMBs, IT service providers and consumers.

During 2009 they used 48\% of their revenue on sales and marketing, thus they commit significant resources to reaching new customers, but when they compare themselves to their largest competitors they state that, while these \q{attract new customers through traditional marketing and sales efforts, [we] have focused first on building a large-scale community of users.} The primary rationale for doing this, according to the company, is that they are competing in a competitive market with low barriers to entry. They state that: \q{We believe our large user base also gives us an advantage over smaller competitors and potential new entrants into the market by making it more expensive for them to gain general market awareness.} Thus, LogMeIn competes with companies using freemium by using freemium themselves.

According to LogMeIn \abbr{CEO} Michael Simon, their free service is \q{the oxygen supply to our business.}

% \todo{Lag tabell fra customers and users-sliden i \url{http://files.shareholder.com/downloads/ABEA-358CAD/1089163783x0x427024/6ddd9123-e051-4e2d-b199-a5ebce29dea0/LogMeIn_Investor_Presentation_Q4_10.pdf}}
% 
% Unpaid sources of demand generation: \q{Direct Advertising Into Our User Community.	We have a large existing community of free users and paying customers. Users of most of our services, including our most popular service, LogMeIn Free, come to our website each time they initiate a new remote access session. We use this opportunity to promote additional premium services to them.}
% 
% Marketing initiative: \q{Social Media Marketing. We participate in online communities such as Twitter, Facebook and YouTube for the purpose of marketing, public relations and customer service. Through these online collaboration sites, we actively engage our users, learn about their wants, and foster word-of-mouth by creating and responding to content about LogMeIn events, promotions, product news and user questions.}
% 
% Marketing initiative: \q{Web-Based Seminars.	We offer free online seminars to current and prospective customers designed to educate them about the benefits of remote access, support and administration, particularly with LogMeIn, and guide them in the use of our services. We often highlight customer success stories and focus the seminar on business problems and key market and IT trends.}
% 
% \q{We believe that our large-scale user base, efficient customer acquisition model and low service delivery costs enable us to compete effectively.}
% 
% \q{LogMeIn Hamachi is offered both as a free service for non-commercial use and as a paid service for commercial use.}
% 
% Bør gå inn og se på tallene og gjøre en analyse av hvor dyre gratisbrukerne er.
% 
% \url{http://www.masshightech.com/stories/2010/06/21/daily2-LogMeIn-sees-stock-and-app-take-off-with-iPad-success.html}
% 
% \url{http://www.investors.com/NewsAndAnalysis/Article/536360/201006041730/Switching-From-Free-To-Fee-.aspx}
% 
% \url{http://bits.blogs.nytimes.com/2009/09/28/logmein-links-to-pcs-and-ford-150s/#more-20149}
\section{Case 2 --- Evernote}

Evernote\footnote{The presentation of Evernote is based on their sharing of information in \citet{libin2010,libin2010b,libin2010c,libin2010d,libin2010e,grove2010,darlin2009,sattersten2010}.} is a suite of software and services designed for notetaking and archiving, based on the \abbr{SaaS} model. Phil Libin, \abbr{CEO} of Evernote, describes it as an \q{external brain} and a \q{universal memory drawer.} The company launched in June, 2008, and has more than 5 million users. Of these users, about 160,000, and thus 3.2\%, are paying users. Evernote has received \$20 million \abbr{USD} in funding from \emph{Sequoia Capital}, one of the most prominent venture capital firms in the world.

The Evernote service has three price plans: free, \$5 a month and \$45 a year. They are, however, very committed to their free option, stating that \q{We want most people using Evernote for free.} Their premium offering is one of Evernote's five revenue models but stands for almost 90\% on revenue.

Evernote does no advertising at all, as \q{our product is our marketing.} This is interesting as their software is \q{not a social app,} as Libin himself states it, are as there are therefore no network externalities. However, according to Libin \q{98\% of our growth comes from people telling their friends}. Thus, Evernote has significant word-of-mouth effects.

For their customers, Evernote have found that the longer they use the service, the more likely they are to upgrade to the paid version. Thus, the longer they remain customers, the more profitable they become. Usually Evernote have seen about 0.5\% of new users becoming paying customers in the first month. In the same first month they usually lose 50-60\% of the new users, and usually an additional 5-10\% the second month. However, from that point on they almost don't lose any customers at all. Looking 29 months in on a group of users that joined the service the same month, more than 20\% have converted to paying. According to Libin, \q{the more memories users store in Evernote, the more invested they become.} In the same vein he has said for freemium to work a company \q{need a product that increases in value over time.}

Based on this gradually increasing conversion rate and as the users of their paid version is \q{growing much faster than our free user base}, Evernote is experiencing an exponential growth in revenue --- more specifically they have experienced month-over-month revenue growths of almost double the month-over-month user growth.

The company has a \$0.25 \abbr{USD} revenue and variable expenses of \$0.09 \abbr{USD} per month per active user, and thus a gross margin of \$0.16 \abbr{USD}. Within these numbers are many interesting points. First of all, as they have a low revenue per active user per month, they need to acquire customers very cheaply. The company in fact does no advertising at all, only relying on organic growth, and primarily word-of-mouth. Secondly, they have low variable expenses. According to Libin, they modeled Evernote to be profitable at a 1\% conversion rate. Thus, the company is built to have a very low variable expense per user. 

Connecting these costs and the fact that Evernote's conversion rate increases with how long they have used the service, Evernote focuses on having a high long-term retention rate. Libin explains this as \q{[the customers] stay a long time, they convert more as they age, and their costs stay flat --- that's what you need to make freemium work.}

Evernote has, according to themselves, built their premium version specifically for \term{power users}, \ie those that very actively use the service, and included several features that are not necessary for most users. An interesting point is that they see an increase in revenue when adding additional premium features, no matter if the converting customer use them or not. Interestingly, this means that conversion from free to premium is not based on use. A possible explanation for this, is that it can be based on the perceived value of the product increasing with the functionality upgrade.

One thing the company states over and over again publicly, is that \q{People pay for what they love.} They are very focused on value creation. An interesting point with regards to this, is that Libin has said that \q{we just haven't had time to get into heavy price theory yet,} and that the primary reason why people upgrade to the premium version is because \q{they love Evernote.} Explaining their pricing, he says that \q{We decided to go with a single, low price for the premium version to keep the decision for users as simple as possible.}

Evernote has not publicly stated their net income, only the fact that they have a positive gross margin, and that it is increasing.

% http://blog.evernote.com/2010/10/19/evernote-raises-20-million-led-by-sequoia-capital/

% The free service is supported by advertising, on both the web interface and in the application. The premium service allows the user to turn off this advertising.
% 
% \url{http://blogs.wsj.com/venturecapital/2009/10/25/sequoia-capital-charges-hard-through-recession/}
% 
% \url{http://www.nytimes.com/2009/08/30/business/30ping.html}:
% 
% \q{Free is not a loss leader, if we can get a small percentage of users to pay we start to make money.}
% 
% \url{http://www.fastcompany.com/magazine/147/next-tech-remember-the-money.html}:
% 
% \url{http://gigaom.com/2010/03/26/case-studies-in-freemium-pandora-dropbox-evernote-automattic-and-mailchimp/}:
% 
% http://www.youtube.com/watch?v=NvQeb-MGXCY
% 
% Fra \url{http://www.youtube.com/watch?v=5dzLIqoq4nM}:
% 
% \begin{items}
%   \item \q{We don't think of free as a loss leader.}
%   \item \q{Multi-platform use is the greatest predictor for user conversion to premium.}
%   \item Bruker eksempel fra mars 2008. Halvparten haller av iløpet av første måned, 5-10\% neste måned, og så er det helt flatt. Retention: 11 000 people.
%   \item \q{Even though the number of active users declines as the cohort ages, the total revenue keeps increasing.}
%   \item \q{We have not focused on the conversion rates}
%   \item First month: 2-3 cent per aktive bruker. 2 år: 70 cents. 
%   \item Low variable cost
%   \item It is a numbers game
% \end{items}
% 
% \url{http://eu.techcrunch.com/2010/12/10/evernote-ceo-phil-libin-at-le-web-tctv/}:
% 
% \url{http://www.mobile-ent.biz/news/39807/LeWeb-10-Evernote-has-160000-paying-users}:
% 
% Playfish co-founder Sebastien de Halleux said at LeWeb this week that Facebook social games tend to get between 1\% and 5\% of players paying for content.
% 
% \url{http://www.youtube.com/watch?v=Oj-ougERrUQ}:
% 
% \begin{items}
%   \item \q{People pay for what they love.}
%   \item The percentage doesn't matter, but the total number that pays matters.
% \end{items}
% 
% \url{http://mashable.com/2010/12/10/evernote-growth/}:
% 
% We've found that the more content we provide, the more engaged our users become, and the more likely they are to tell a friend and convert to Evernote Premium.
% 
% \url{http://toddsattersten.com/2010/02/fixed-to-flexible-interview-with-evernote-ceo-phil-libin.html}:
% 
% The gross margin comes out to about 53\%. The gross margin increases every month because revenue per user grows (conversion rates go up because long-time users are more likely to convert) while variable expenses per user decline (Moore's law + efficiencies of scale). 
% 
% We launched the service into closed beta in February of 2008. Gross margin went positive in January 2009.
% 
% 35\% active users (defined as being active a given month)
% 2,335,676 total users
% 41,598 premium users
% 38972 users per server
% 4 sysadmins
% \$68,641 - Total variable expenses (hardware + software + hosting + network + operations staff + support staff)
% \$145,000 - Total revenue from active premium users
% 
% From this we can deduce some interesting numbers:
% * free/paying ratio (all users) - 1,7\%
% * free/paying ratio (active users) - 2.2\%
% * revenue per user (all users) - \$0,06
% * revenue per active premium user - \$3,5
% * Variable expense per user - \$0,29
% * 38927 users per server
% * 583919 users per sysadmin
% 
% I particularly find the \$0,06 revenue per user interesting, since it gives a rough number to work with if you're doing a freemium website. Eg. if you need \$10.000 in monthly revenue to break even you will need 166.666 users.
% 
% \q{Invest heavily in the product; focusing on things that will make your customers love you and things that will keep your variable costs low when you scale. Make your product free so that you don't have to pay for traditional marketing.}
% 
% \q{Slowly introduce paid features but always keep the \q{main} product free. Get to positive gross margins}
% 
% The real medium-term cost savings comes from efficiencies of scale. For example, we currently have about 60 servers in the data center and four operations guys to maintain them (both included in the variable costs for gross margin calculations). When we get to 600 servers, we won't need 30 ops guys, probably 10 will be enough.
% 
% Here are three things I think about [when rolling out a freemium model]:
% 
% 1. Make a product that a billion people will fall in love with and use for the rest of their lives.
% 2. Make it easy for a single-digit percentage of them to pay you a few bucks a month once in a while.
% 3. Make sure your variable costs are low enough that you can make a mountain of profit if you get \#1 and \#2 right.
% 
% If you can't see how you'll do all three things, go with another business model.
% 
% \url{http://www.bbc.co.uk/blogs/technology/2009/12/freemium_can_spotify_learn_fro.html}:
% 
% \q{We've decided to experiment with full transparency on the numbers}
% 
% the few adverts that come with the free service only produce negligible revenue
% 
% \url{http://vimeo.com/11932184}:
% 
% Great long term retention rate, product that increases in value over time, low variable costs.
\section{Case 3 --- MailChimp}

MailChimp\footnote{The presentation of MailChimp is based on their sharing of information in \citet{chestnut2010,chestnut2010b,chestnut2010c}.} delivers an email marketing service to design, send, and track HTML email campaigns. The company launched in 2001, and was a premium-only service for eight years, until they changed to freemium in September, 2009. At that point they had 85,000 users, which grew by five times to 450,000 users in the following 12 months. They are now adding 30,000 new users each month, of which more than 13\% are paying customers.

In the 12 months from September 2009 when they chose to go freemium, MailChimp increased their number of paying customers with 150\% and their profit with more than 650\%. According to the company the primary reason for their increased profit, is that their cost of customer acquisition had dropped considerable. In the last quarter alone the cost fell by 8\%. 

From April 2010 through August 2010 the company saw an increase on their largest pricing plan from 12\% to 20\% of paying customers. These 20\% account for 65\% of MailChimp's total revenue, up from 48\% in April. Thus, a very important growth factor during the 12 months of freemium was on their most expensive plan --- a growth that was considerably cheaper to achieve with freemium.

Ben Chestnut, the \abbr{CEO} of MailChimp, explains that there were two reason for changing to a freemium model. First of all they had an affordable self-serve product which was scalable and had good abuse prevention, and secondly cloud computing made \q{all [this] even cheaper.} Together this means that they had an highly automated system that scaled well, and that was becoming cheaper and cheaper for them to operate and manage. Based on this, they saw an opportunity to grow by using freemium.

However, abuse was one of the elements that troubled the company after going freemium. They saw an increase of more than 350\% in monthly abuse related issues. Ben Chestnut explained that solving this problem manually would \q{would at least have taken 30 human beings,} which is almost as many as the 38 that are currently employed at MailChimp. This is one of the problems they have had to tackle by automation to be able to stay with freemium.

An interesting aspect of MailChimp's change to freemium, is, as Ben Chestnut states, that \q{it was also a way to thank our customers.} They have seen many users change to a free model, and they are now doubling the free plan and are expecting to see even more convert down to free --- but at the same time, see a considerable increase of their premium plans.

% \url{http://www.mailchimp.com/blog/going-freemium-one-year-later/}:
% 
% \begin{items}
%   \item We're delivering roughly 700 million emails per month
%   \item Here’s another piece of history. Ever since inception, I’ve been fascinated with the art and science of pricing. I’ve tinkered with pay-as-you-go and monthly plans for \$9, \$9.99, \$25, \$49, \$99.99 and so on. We’ve changed our pricing models at least a half-dozen times throughout the years, and along the way we tracked profitability, changes in order volume, how many people downgraded when we reduced prices, how many refunds were given, etc. We’re sitting on tons of pricing data. When we launched our freemium plan in 2009, you betcha we used that data to see what would happen if we cannibalized our \$15 plan.
%   \item Build up that \q{1} before you chase the \q{10.} After you've got your \q{1} all set, use VCs to help you chase after that \q{10} (if you must). That's my personal opinion.
%   \item I don’t want anybody to think we launched freemium without a lot of careful thought and planning. And ever since launching, we’ve monitored our stats, and learned more about our freemium users, their impact on our business, and ultimately, used that data to decide whether or not we should double it this year.
% \end{items}
% 
% With 20\% of their user base (450k) on 10,000+ subscribers per month (\$150 pcm) I make that \$13,500,000 per month. And with that being 65\% of their total revenue, they are making a cool \$20m a mth. Not bad, not bad at all.
% 
% So, at least in my case, Freemium works best for products that either directly or indirectly contribute to making money.
% 
% I'd argue that freemium works best when the \q{premium} part has value for the user. But, making money (i.e. directly contributing to revenue/profit) is just one way to have value. Saving time is another.
% 
% I'd also rather the user tries out my app rather than the competition's
% 
% \url{http://www.youtube.com/watch?v=29j0kyCi0dM}:
% 
% \begin{items}
%   \item Around for 10 years
%   \item \q{Because of the cloud [...] we noticed that everything was getting cheaper.}
%   \item More than 200\% revenue growth
%   \item +354\% abuse related issues per month.
%   \item \q{You are going to have abuse related events if you are successful}
%   \item 38 person company
%   \item Automated handling of abuse related events (Koble mot Osterwalder)
%   \item \q{It would at least have taken 30 human beings to do this}
% \end{items}
\section{Case 4 --- Helpstream}

Helpstream\footnote{The presentation of Helpstream is based on their sharing of information in \citet{warfield2010,warfield2010b,warfield2010c}.} delivered a social customer service and relationship management (\abbr{CRM}) system, based on \abbr{SaaS}, which they explained as \q{community-driven customer service} --- customers helping each other resolve issues and answer questions. The company was founded in 2004, and went out of business in March 2010. The company had about \$8.6 million \abbr{USD} in funding.

The company was primarily \abbr{B2B} oriented,\footnote{They had consumer \emph{users} which increased the value for these businesses by being those who contributed to the customer service, but these could not upgrade to a premium version.} and at the end they had 240 users, of which 40 was on the premium service. There was, however, more than 500,000 registered users. Helpstream built an underlying technology that enabled them to achieve a cost of \$0.05 USD, 5 cents, per registered user per year, which Bob Warfield, the former \abbr{CEO} of Helpstream, states as \q{extremely low.} 

An interesting aspect of Helpstream, is that they had problems with the registered businesses who used freemium. Warfield states that \q{the biggest problem was that it was not attractive to the \emph{right kind} of customers,} as freemium was \q{self-selecting customers who were most likely to be willing to pay, well, nothing!} In addition, even though they had 500,000 users, these were not customer prospects for Helpstream, and all used the service for free. Thus, the company was not able to generate enough customers to survive.

Another of the company's problem, was that they had internal problems relating to the use of freemium, especially from the Sales and Marketing departments. The problem was Sales saw freemium as something that took away from their commission, and that they then became \q{actively hostile} --- Sales saw freemium as competition. Explaining the problem this created in the organization, Warfield states that \q{it became almost impossible to talk about the Freemium without polarizing armed camps, which eventually made it a sacred cow.}

The primary value Helpstream gained from using freemium was that \q{the free users gave us a tremendous amount of feedback and testing.} It also enabled the company to show momentum in terms of the number of customers. Despite this, Warfield recommend using a trial for \q{a company like Helpstream.} Looking back, he see two elements that could have helped the company; lowering the adoption friction and reducing the transaction friction --- thus, making it easier to use and easier to pay, especially as the software once registered \q{was sterile and did not give them an ability to see it in the context of their intended use.}

% \url{http://smoothspan.wordpress.com/2010/03/22/my-startup-track-record/}
% \url{http://sanjose.bizjournals.com/sanjose/stories/2008/02/25/daily64.html}
% \url{http://smoothspan.wordpress.com/2010/03/25/freemiums-for-saas/}:
% 
% \begin{items}
%   \item [The original CEO] wanted to price everywhere along the demand curve so as not to leave a flank exposed at some price point.
%   \item So, prospects would get a free workspace and play with it a bit, but it was sterile and did not give them an ability to see it in the context of their intended use. 
%   \item I looked at the usage statistics to determine who was deriving real value.  A hospital using the Freemium to deliver an IT Help Desk to hundreds of people was clearly deriving real value.  Based on these statistics, I set limits on the Freemium.
% \end{items}
% 
% \url{http://smoothspan.wordpress.com/2010/04/05/references-and-pricing-for-saas-startups/}:
% 
% \begin{items}
%   \item The first task when selling business software is getting anyone to pay anything for it and then be willing to talk about it so you have references.
%   \item Give it away in order to get a list of referencible customers started.  It’s very hard to sell business software without some references.
%   \item The initial references are going to come from networking.
%   \item This isn’t about extracting revenue.  This is about building credibility and learning something about the world’s reaction to you product. 
%   \item At Helpstream we started out at about half the cost of the low-cost players.  By the end of my tenure we were at half the cost of the highest cost players, and that by dint of discounts and much lower friction (more on lowering friction in a future post) in terms of professional services to install.
%   \item Most SaaS companies have way too complicated pricing.
%   \item If billing is a manual process, that’s when it becomes a hassle.
% \end{items}
% 
% \url{http://smoothspan.wordpress.com/2010/10/04/a-pair-of-killer-pricing-articles-for-startups/}:
% 
% \begin{items}
%   \item Think of Freemiums and Low Prices as Marketing.
%   \item The more value you recieve, the more you should pay.
%   \item It’s a beautiful thing to outflank the competition with a qualitatively different pricing model.  Qualitative means your pricing isn’t just cheaper, but it works differently.
%   \item Get a good sense of how long it should take them to evaluate your product, get all the ducks in a row on their side to make a purchase, and then use that figure, whatever it turns out to be.  Getting all the players in an org to agree on a Social CRM system in 14 days was never going to happen for Helpstream.
% \end{items}
% 
% Det med å gi det gratis til \q{feil} brukere, er det samme Sixteen Ventures skriver om Ning\footnote{\url{http://sixteenventures.com/blog/exploring-ning-post-freemium-pricing-page.html}}.