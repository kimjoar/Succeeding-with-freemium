\section{Free}
\label{section:free}

When talking about free, there is a distinction between \term{gratis} and \term{libre} --- which can be thought of as \q{free as in free beer} versus \q{free as in free speech,} respectively.\footnote{This is how Richard Stallman, an American software freedom activist and computer programmer, differentiates software that is free to use and software that is freely available for users to change and extend \citep{stallman2010}.} When talking about free in this paper, the focus is on gratis, not libre.

\newthought{In traditional economic theory} there is an inverse relationship between price and quantity demanded \citep{osullivan2003}, but what happens when the price becomes zero? According to \citet{shampanier2007} people perceive the benefits associated with free products as higher. Thus, decreasing the price to zero increases perceived value considerable more --- people overreact to free products. \citet{shampanier2007} also saw a decrease in the demand for the more expensive product when including a free offer. They termed these two findings the \term{zero-price effect.} Driving this effect can be that the decision to choose a free product is a much simpler decision \citep{tversky1992,luce1998,iyengar2000,diederich2003}.

\citet{haruvy2001} found that, in order to not cannibalize the premium offer, the quality of the free offer must be sufficiently low and the price of the premium offer must not be too high --- but, importantly, the free offer must not be of too low quality; it must induce customers to use it.

According to \citet{shampanier2007} there is still much additional work needed to properly understand the complexities of gratis products. However, the research thus far on free software can be broadly categorized into three categories \citep{jiang2010}:

\begin{enum}
  \citerm{Network externalities}{haruvy1998,haruvy2001,gallaugher1999,gallaugher2002,chakravarty2006,cheng2010,parker2005}, in which free adopters increase future adopters' valuation of a product.
  \citerm{Demonstration effects}{manica2003,faugere2007,hui2008,cheng2008,cheng2010}, in which users can try the software before buying.
  \citerm{Word-of-mouth effects}{jiang2009,jiang2010}, in which free software adopters help speed up the diffusion of a new product.
\end{enum}

\subsection{Network externalities}
\label{section:free:networkexternalities}

When a network externality is present, the value of a product or service increases as more people use it \citep{katz1985} --- the classic example being telephony, as having a telephone is only valuable if there are other people with compatible telephones. The value can be seen in regards two eponymous laws:
\begin{inparaenum}[\itshape 1\upshape)]
  \item \term{Metcalfe's Law}, which states that the value of a network is proportional to the square of the number of connected users of the system \citep{metcalfe1995,shapiro1999,hendler2008}; and
  \item \term{Reed's Law}, which states that the utility of large networks, particularly social networks, can scale exponentially with the size of the network \citep{reed2001}. He call these \term{group-forming networks}, wherein network members can create and maintain communicating groups.
\end{inparaenum} 

One of the problems with distributing free versions of a product, is that it can cannibalize sales of the premium version; thereby lowering the profits for the firm \citep{haruvy1998}. On the other hand, the price users are willing to pay is, in part and \term{ceteris paribus}, determined by the number of users in the network to which the product belongs \citep{brynjolfsson1996}, and establishing this initial network is simpler when giving away the product for free \citep{haruvy2001}.

\newthought{As it is advantageous} to achieve a significant share of the market quickly \citep{brynjolfsson1996,katz1992}, the initial purchase should be made as easy as possible. This is especially important since the network benefits are lower for the early adopters. \citet{lee2003} divided customers into two types: power users and light users. Power users are less sensitive to compatibility, \eg lock-in, than light users, and also assumed to be keener to new technologies.\footnote{Power users can be seen as what \citet{moore2002} categorize as \term{technology enthusiasts} or \term{visionaries}, and light users as \term{pragmatists} or \term{conservatives}.} Thus, segmenting the customers such that the initial customers are power users, can make their initial purchase simpler.

\newthought{Two-sided markets}, also called two-sided networks, are economic platforms having two distinct user groups that provide each other with network benefits \citep{parker2005,rochet2006}, \eg credit cards (cardholders and consumers), hospitals (patients and doctors), and Google (searchers and marketers).

According to \citet{rysman2009}, pricing and openness are the two most important strategies for a potential platform. With regards to pricing, it is not only dependent on the demand and costs that consumers bring to one side of the market, but also how their participation affects participation and profit on the other side \citep{rysman2009}. Thus, we see that prices depend on demand elasticities and marginal costs on each side \citep{rochet2003,rochet2006,weyl2009}, which means that lower prices on one side leads to higher prices or more participation on the other side.

What \citet{parker2005} finds, is that profit can be maximized even when pricing below marginal cost in the absence of competition, \eg the product can be rationally given away for free into perpetuity, even when not competing. The key is is low marginal costs, not non-rivalry. The reason for this potentially being profit-maximizing is that increased demand in a the premium goods market more than covers the cost of investment in the free goods market \citep{parker2005,rysman2009}. With competition, the sum of prices on the two sides of the market, \ie \term{price level}, decreases \citep{weyl2006}.

A market where there is externalities between the users of the free version and the users of the premium version, can be seen as a two-sided market \citep{parker2005}. The segment that contributes more to demand for the other is the market to provide with free goods \citep{parker2005}.

\subsection{Demonstration}
\label{section:free:demonstration}

Compared to network effects, product demonstrations play on the intrinsic features of the product, not the extrinsic. According to \citet{faugere2007} a primary purpose of offering a free sample is to increase sales by providing first-hand experience for potential buyers. Being able to try a product before buying has been shown to play a significant role in the adoption of information technologies \citep{agarwal1997}, and in their study of the market for Web server software \citet{gallaugher2002} found that trial versions were associated with price premiums.

As for network externalities, \citet{gallaugher2002} found that being able to try the product before buying it help seed the initial market.

From the typology discussion in \sectionref{context:freepaid}, we see that all freemium products or services have demonstration effects, however to different degrees. To exemplify this, we see that distinguishing by customer class there might be demonstration effects for consumers, but not for companies. 

\subsection{Word-of-mouth}
\label{section:free:wom}

\citet{jiang2009} found that even if other benefits do not exist, \ie network externalities or demonstrations, a company can still benefit from giving away fully functioning software as the free adopters help speed up the diffusion process, \eg giving away software the first month it is in sales. The different methods free adopters can use to help speed up the adoption of the product is verbally, \eg spreading information about its availability and quality, or nonverbally, \eg through others seeing the product in use, peer pressure, or social influence --- together these are referred to as a \term{word-of-mouth effect} \citep{jiang2009}.

As for both network externalities and demonstrations, the influence of word-of-mouth means that companies can increase their profit when they have a free offer \citep{jiang2009}.

However, an important consideration with regards to \citet{jiang2009} is that their study focuses on free software that is offered without any restrictions, while for freemium we would either see time or functionality limitations. No study was found that discussed this more specifically for freemium.