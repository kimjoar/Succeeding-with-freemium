\chapter{Discussion}
\label{chapter:discussion}

This section analyses and discuss the empirical findings, based on the earlier presented theory on business models, information goods and free. This chapter is structured around how and why companies have succeeded and failed with freemium, and from the discussion a set of propositions are created.

First and foremost, this paper examine three companies that are successfully basing their business model on freemium and showing significant positive results from this. Thus, this indicate that it is possible to create a viable business model based on freemium; thereby suggesting that freemium is a potential choice when designing a business model. According to \citet{teece2010}, it is still uncertain as to whether or not business models based on freemium works, the findings in this paper, however, indicate that they can work.\footnote{The sustainability of this success is, however, outside the scope of this paper.} On the other side, this paper also show that success is not guaranteed and that there are considerable difficulties when using freemium. From this the following can be proposed.

\proposition{It is possible to design viable business models based on freemium.}

\section{Creating Value}

All four case companies focus on value creation, delivery and capture, however, to different degrees and with different focus. Evernote is the case company which is most focused on creating value for customers among the four, stating that \q{people pay for what they love,} and that this is what they are continuously focused on. According to Evernote the essence of value creation is to create a product that increase the perceived value over time.

However, an interesting aspect appears with regards to Helpstream, the only case company that failed. Bob Warfield, who took over as \abbr{CEO} for the company, has explained how their free version \q{was sterile and did not give [the registered companies] an ability to see it in the context of their intended used.} They were not able to utilize the demonstration effects of the free version, and thereby were not able to create enough value for the free users to convert them to premium. As we saw in \sectionref{free:demonstration}, \citet{faugere2007} explains that a primary purpose of offering a free sample is to increase sales by providing first-hand experience for potential buyers. As Helpstream was not able to do this, their company had problems converting free users. All the other three case companies have been successful at this. From this we can propose:

\proposition{The free version must convince a large enough subset of users to convert to paying customers by conveying the added value of the premium version.}

Another interesting aspect of value creation, is the extent to which word-of-mouth was a focus among the case companies. Evernote goes as far as stating that \q{98\% of our growth comes from people telling their friends.} LogMeIn see word-of-mouth as on of the most important parts of their customer acquisition process, which they see as a key competitive advantage. Thus, an important aspect of freemium seems to be users telling others about a product or service, and therefore the product must be created to achieve this. Helpstream was not able to do this to a large degree, and was therefore not able to achieve a strong growth in companies registered. An aspect of this can that they only charged for \abbr{B2B} and not \abbr{B2C}, however, discussing this is outside the scope of this paper.

When discussing their use of free, all four companies focus on using freemium as a marketing tool to reach more customers and to lower the cost of customer acquisition. According to Helpstream \abbr{CEO} Warfield a company should think of freemium as marketing. In the same vain, Libin says \q{Make your product free so that you don't have to pay for traditional marketing.} Similar statements come from the other two companies, \eg LogMeIn's statement \q{We believe our free products [\ldots] help to generate word-of-mouth referrals}. And it is the word-of-mouth effects that are primarily mentioned in all four cases related to using freemium as a marketing tool. 

\proposition{Strong word-of-mouth effects is vital for success with freemium.}

Among the four case companies only Helpstream is inherently social, and therefore expected to have significant network externalities. The company employed a two-sided market, as described in \sectionref{free:networkexternalities}, with regulars users on one side of the network and businesses on the other side, which provide each other with network benefits. However, Helpstream was not able to obtain sufficient income from the paying side, the businesses. An interesting aspect with regards to the theory, is that the theory only discuss giving the product away for free on one side of the market, not at both. As \citet{rysman2009} states, pricing is not just dependent on demand and cost on one side, but how one side's participation affects participation and profit on the other side. Thus, lower prices on one side leads to higher prices or more participation on the other side. For Helpstream we see that they were able to attract regular users, having more than 500,000, but these users were not able to attract more businesses on the other side. This demands higher prices, but as Helpstream has noted, they \q{started out at about half the cost of the low-cost players.} They saw this problem, and increased their prices, but were still not able to survive. From this we see the importance of understanding the dynamics of network externalities and two-sided markets, and how it plays out among the company's customers.

\section{Delivering Value}

Looking at delivering value, we see that all four companies have a strong focus on this. Helpstream was able to deliver its products for \$0.05 per user per year, Evernote was at \$0.09 per user per month. LogMeIn saw their ability to deliver their product at a very low cost as a key competitive advantage, while MailChimp chose freemium specifically because they saw their platform could handle giving away their product for free to most of their users. As freemium entail giving away potentially most of a product or service for free, delivering this value must be very cheap. Within these costs are amongst others data center operations, customer support operations, hosting feeds, maintenance, and software licenses. From these four companies we can see that delivering their product or service cheaply is a necessity, but not something that will by itself make a company succeed with freemium, as we see from Helpstream.

\proposition{As a precondition to choose freemium a company must have a low marginal cost, at least on the free version.}

Another aspect of value delivery, is the potential customers to deliver to. With conversion rates in the single or low double digits as we have seen in this paper, most users will be using the free version. Thus, an important metric for freemium based companies is the size of the addressable market. However, to find the total potential for paying customers, it is also essential to consider how many within the addressable market that can be reached, the percentage that registers, and the percentage that converts to premium. Again, we see the importance of continuosly measuring when using freemium.

\proposition{Freemium depends on a large addressable market.}

For delivering value, an important choice is how the company reaches its customers. This is an aspect that Helpstream had considerable problems with. For them freemium created \q{polarizing armed camps} within the company. The primary reason for this was that the sales department saw freemium as competition. LogMeIn, the other \abbr{B2B} oriented company among the cases, also has a sales department, but have not had the same problem. However, differently than Helpstream, LogMeIn only used freemium to reach consumers and get them to use their free version. Thus, the sales department, which is primarily used to sell to other businesses, will not see freemium as competition. From this we see the importance of understanding both the potential internal and external problems of choosing freemium.

\section{Capturing Value}

For capturing value, I expected based on the theory to see discussions among the case companies on pricing, \eg as discussed in \sectionref{context:freepaid} and \sectionref{informationgoods:versioning}. However, none of the four have talked much specifically about their pricing with regards to freemium. For all the four, there seems to be a focus on measuring and tracking as much as possible, to have a clear understanding of what prices are deemed necessary to make a profit on those who pay, and of course, how much they are able to get their premium customers to pay. As we have discussed earlier, \citet{varian1997} states that a fundamental problem for information goods is to set the prices such that those users that are able and willing to pay high prices do so --- and for the four companies segmentation seemed more important than pricing in itself. According to Phil Libin, the \abbr{CEO} of Evernote, \q{the percentage doesn't matter, but the total number that pays matters.}

Warfield explains that for Helpstream setting the boundary between free and paid was one of the elements where they failed with freemium, and where he, when he took over as \abbr{CEO}, used their statistics to set limits on the freemium in order to try to get more people over at their premium offerings. Thus, we see that to price the product it is very important to understand the customers and the cost of doing business.

Libin has called freemium \q{a numbers game.} As freemium entail possibly giving away the product or service to most users for free --- which we also see in all our four case companies where there are conversion rates in the single digit or low double digit percents --- understanding the costs associated with both bringing in users, serving them the product or service, converting them, keeping them in the system, and most importantly their lifetime value, are essential metrics. As freemium include a large amount of users not paying, it is less transparent than a \q{regular model,} where each user pays for him- or herself. As discussed in \sectionref{context:what}, the quid pro quo in freemium is not about money for the free users, but about using the free users to make money. From this we can propose the following.

\proposition{When using freemium it is important to track and measure the costs and revenues from the product or service.}

An important lesson that can be learned from tracking and measuring, is when, as discussed in \sectionref{free}, the quality of the free offer is not low enough, and therefore cannibalize the premium offer. In addition this tracking and measuring activity is in line with \sectionref{businessmodels:time}, discussing the importance of experimenting with a business model. As \citet{sosna2010} notes, even incremental changes to a business model can have significant effect of its success. In the same vain, Drew Houston, \abbr{CEO} of Dropbox, another freemium based company, has stated that \q{It’s all about finding things in the margins --- lots of little things rather than one key thing} \citep{gannes2010}.

One of the most discussed elements of freemium in the public sphere, is the conversion rate from free to premium. However, non of the four companies put much focus directly on this number. Evernote discusses conversion rates deeply, but to a larger extent in the context of the degree they are able to keep the users in their system, \ie the retention rate. The key focus for them is their ability to convert the customers that \emph{stay} in the system. Thus, the focus is on the lifetime value of the customer, not the conversion rate per se. We see a similar story from LogMeIn, which don't discuss conversion rates, but rather discusses the average lifetime value of each customer. MailChimp in the same vein talks about how their lifetime value has increased as customer acquisition has become cheaper and more users are converting to their most expensive pricing plan. Having a high conversion rate, but a low retention rate, gives a low customer lifetime value. Based on this, we can propose the following. 

\proposition{To succeed with freemium, the key metric is customer lifetime value.}

Looking at customer segmentation, we see that Helpstream had problems as they were not able to attract \q{the right kind of customers} for the free version. Thus, they were not able to decrease the cost of customer acquisition and increase the lifetime value of their customers. An vital problem was that Helpstream attracted customers that were not likely to convert from free. For all the three other customers, we see that they were able to attract customers that can be converted to the premium version, and which thus decreased their costs and increased their revenue. Thus, to capture value, it is key to target the right customers with the right product at the right price, which again show the importance of measuring and tracking when using freemium.

One of MailChimp's key activities as en email marketing service is handling abuse. Without automating this process, they have stated that it \q{would at least have taken 30 human beings,} which would have been too expensive to sustain their freemium model. Thus, automation was vital for their success.

\proposition{The company must be able to automate the most manually demanding activities related to their free users to succeed with freemium.}