\section{Information Goods}

% \q{Information wants to be free. Information also wants to be expensive. \ldots That tension will not go away.} --- \citet[\p{202}]{brand1988}

An \term{information good} is anything that can be digitized \citep{varian1999}. An important feature of information goods is that they are expensive to produce and cheap to reproduce \citep{varian1995,varian1997}. Pricing the product according to the marginal cost will not make sense in this context, since the marginal cost will be zero (or negligible) \citep{varian1997,shapiro1999,mahadevan2000}. Rather, it should be based on value \citep{varian1997}. On the other hand, \citet{anderson2009} argues that \q{practically everything Web technology touches} will end up as free for consumers, as the marginal costs are approaching zero and prices are approaching the marginal cost since \q{there's never been a more competitive market than the Internet.}

Based on the arguments in \citet{varian1995} it is likely that the prediction of free products on the Internet in \citet{anderson2009} only yields for \term{purely competitive markets}, \ie when there are \q{several} producers of an identical commodity. On the other hand, it is not necessarily true\footnote{But it might be true. There are still no available academic research specifically relating to this statement, and \citet{varian1995} was written before the implications of the Internet where clear.} for markets with \term{monopolistic competition}, \ie when there are several somewhat different products some of which are close substitutes. The focus in this paper is on markets with monopolistic competition, thus when not all products and services necessarily are pressured towards being free.

\subsection{Versioning}
\label{section:informationgoods:versioning}

Versioning, \ie differential pricing based on quality discrimination, lets consumers \term{self-select} into different groups according to their willingness to pay \citep{varian1997}. According to \citet{varian1997} the fundamental problem is to set the prices such that those users that are able and willing to pay high prices do so. Thus, this is the essential quandary with regards to customer segmentation, one of the building blocks in the Business Model Canvas, as discussed in \sectionref{businessmodels:canvas}.

Determining a users willingness to pay is difficult as consumers are not eager to reveal their true willingness to pay \citep{varian1995}. As such, pricing needs to be based on something that is correlated with willingness to pay, \eg observable characteristics, such as memberships in certain social or demographic groups; or unobservable characteristics, such as the quality of the choice the consumer purchases. An interesting aspect of the Internet and freemium with regards to this, is how easy it has become to experiment with pricing plans to find out how much users are willing to pay.

% \newthought{Looking specifically} at pricing, there are several interesting aspect. According to \citet{smith1995} \q{adding a premium product to the product line may not necessarily result in overwhelming sales of the premium product itself. It does, however, enhance buyers' perceptions of lower-priced products in the product line and influence low-end buyers to trade up to higher-priced models.} In psychology, this is termed \term{extremeness aversion}, and with regards to pricing several authors have researched this phenomena \citep[\eg][]{simonson1992,chernev2004,tversky1993,huber1982}. 

% In addition, \citet{varian1995} state that decreasing the quality of the low-quality bundle, enable the seller to charge more for the high-quality bundle.

% \todo{{\em Ha med noe om profit-maximizing i forbindelse med versioning}}
% \todo{{\em Rett antall versjoner. Kan bli for mange. Se f.eks. på \citet{tversky1992,luce1998,iyengar2000}}}