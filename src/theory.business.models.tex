\section{Business Models}
\label{section:businessmodels}

The business model affect an organization's potential for creating and capturing value \citep{amit2001}, and is a key component of organizational success \citep{badenfuller2010,chesbrough2010}. Additionally, the choice or design of a business model can be a source of competitive advantage \citep{christensen2001,gambardella2010,teece2010,malone2006,masanell2010,zott2008}. \citet[\p{12}]{chesbrough2007} goes as far as claiming that \q{a better business model will beat a better idea or technology.}

With the advent of the Internet, business models have become extensively discussed \citep{rajala2003,zott2008,demil2010,ghaziani2005,sosna2010,teece2010},\footnote{A limitation with regards to the research on business models is that only few contributions have appeared in top journals \citep{zott2010b}.} and the Internet has led entirely new business models to become viable \citep{petrovic01,geoffrion2003}. Research on business models can be organized around two complementary streams \citep{zott2010b}: 
\begin{enump}
  \item Describing generic business models and providing typologies, \eg the \emph{auction model} and the \emph{Dell model}; and
  \item Describing components of business models, \eg revenue streams and customer segments.
\end{enump}. In this paper the latter is the most relevant, and will thus be the primary focus of this review.

% \begin{enump}
%   \item Describing generic business models and providing typologies \citep[\eg][]{timmers1998,tapscott2000,rappa2010,weill2001}, \eg the \emph{auction model} and the \emph{Dell model}; and
%   \item Describing components of business models \citep[\eg][]{mahadevan2000,alt2001,osterwalder2004}, \eg revenue streams and customer segments.
% \end{enump}

Even though there have been much focus and research, there is still no generally accepted definition of the business model concept \citep[\eg][]{shafer2005,pateli2004,osterwalder2005,teece2010,badenfuller2010models,zott2010b}, for example \citet[\p{74}]{porter2001} argues that the definition is \q{murky at best}. 

\subsection{What is a business model?}
\label{section:businessmodels:what}

As there are no generally accepted definition of a business model, I will very briefly review the most cited definitions, compare them, and then choose a definition for the remainder of this paper.

\citet{timmers1998} had the initial definition of a business model with respect to the Internet.\footnote{Which he termed \q{electronic markets.}} He focused on the flow of products, information, and money, and defines a business model as
\begin{enump}
  \item an architecture for the product, service and information flows, including a description of the various business actors and their roles; 
  \item a description of the potential benefits for the various business actors; and 
  \item a description of the sources of revenues.
\end{enump} Similarly \citet[\p{34}]{weill2001} defined a business model as a \q{description of the roles and relationships among a firm's consumers, customers, allies, and suppliers that identifies the major flows of product, information, and money, and the major benefits to participants.} From both definitions we see a focus on identifying the primary components of a business model and their possible interrelationships. The only evident difference between these definitions is that \citet{timmers1998} only mention \q{business actors} without specifying who these are, while \citet{weill2001} are far more specific. 

However, missing from both \citet{weill2001} and \citet{timmers1998}, is a focus on customer value, a central element of business model definitions by many prominent researchers. For \citet{linder2000} a business model's focal point is on the core logic for creating value, and similarly \citet[\p{512}]{amit2001} defines a business model as \q{the content, structure, and governance of transactions designed so as to create value through the exploitation of business opportunities.} Thus, both are focused on creating value, but interestingly \citet{amit2001} also include business opportunities, which all other reviewed definitions do not mention. 

Where the latter two definitions focus on the creation of value, \citet{chesbrough2002} is concerned with \q{realization of economic value.} Thus, this definition is primarily focused on a business being able to generate a profit, while the former value oriented definitions have a broader view on value. \citet[\p{179}]{teece2010}, as \citet{chesbrough2002}, focuses on this economic value, by including \q{a viable structure of revenue and costs for the enterprise} in his definition. However, a contrast between these definitions is that \citet{teece2010} specify that this economic value is a result of delivering a value to customers. Building on, amongst others, these definitions \citet{osterwalder2010} define, as explained thoroughly in \citet{osterwalder2005}, a business model as \q{the rationale for how an organization creates, delivers and captures value.} As we see, this completes the picture from these value oriented business models by including all three elements --- value creation, delivery and capture --- that are mentioned with varying degree in the other definitions.

\citet{gordijn2002} focused on graphically representing the business model, and looked at it as a conceptualization of a business idea. The focus in his methodology was showing which parties exchanged things of economic value with whom, and what they expected in return. Another definition primarily focused on revenue, is \citet{rappa2010} who sees a business model as \q{the method of doing business by which a company can generate revenue to sustain itself.}\footnote{Which is in line with the notion of viable in this paper.} While not specifically focused on revenue, the definition in \citet{zott2008} is similar to these definitions in their focus on a firm's transactions with customers, partners and vendors.

More abstract than all the other mentioned business model definitions, is \citet[\p{4}]{magretta2002}, who see a business model as a \q{story that explains how an enterprise works.} In explaining this definition, however, he introduce several of the elements elaborated by other definitions, such as creation and delivery of value, and how to make a profit.

\newthought{Based on the} definitions we have discussed so far, we see that the research has explored many different directions, and that the definitions vary with regards to degree of abstractness and completeness. There are three aspects that summarizes most business model definitions, namely their focus on creating, delivering and capturing value \citep{osterwalder2005,itami2010,sosna2010,teece2010,zott2010b}. Based on this, the definition in \citet{osterwalder2010} will be used in this paper. This definition include these three elements specifically, in addition to the further elaboration of the definition include all the components found by \citet{pateli2004} to be apparent in most research on the atomic elements of business models. 

Thus, the implications for this paper is that looking at freemium in the context of business models entails an examination of value creation, delivery and capture, in addition to the the atomic elements of this frameworks, which will be discussed in \sectionref{businessmodels:canvas}.

\newthought{An important note} on business models, is that business models and strategy are distinct constructs \citep{chesbrough2002,magretta2002,seddon2003,zott2008,teece2010,zott2010b}: While the focus in a business model is on a firm's exchanges with others, strategy is more about the firm's activities and actions in light of competition.\footnote{There are many intricacies to the discussion of strategy versus business models, but a more formal clarification and discussion is outside the scope of this paper. The point here is only to clarify that they are not equivalent constructs.} As \citet{porter1996} describes, \q{the essence of strategy is choosing to perform activities differently than rivals do.} Thus, when looking at freemium in this paper, competition will not be considered.

\subsection{Business models \oldand the aspect of time}
\label{section:businessmodels:time}

\q{Innovation distinguishes between a leader and a follower.} --- Steve Jobs, CEO and co-founder of Apple, Inc.

Business models have mainly been researched from a static point of view \citep{osterwalder2005}, but recently the aspect of time has been incorporated into the business model research \citep{sosna2010}. According to \citet{chesbrough2002} and \citet{chesbrough2010} successful businesses are those that are able to experiment with their business model, and \citet{linder2000} suggest that firms should actively pursue business model changes. \citet{sosna2010} notes that even incremental changes to a business model can have significant effect of its success, and states that business model experimentation can be strongly beneficial for a firm to \q{ensure sustainable value creation, robustness and scalability} \citep[\p{400}]{sosna2010}.

Interviewing 765 leaders in corporate and public sectors, \citet{rometti2006} found that the financially best performing firms invest twice as much as the under-performers on business model innovation. \citet{giesen2007} argue that business model innovation comprises three types, which, separately or combined, can lead to increased financial performance:

\begin{enum}
  \iterm{Industry} Innovating the value chain, \eg horizontal moves, such as Apple's move from computers to music, ads and phones; moving into new industries; or developing entirely new industries or industry segments.
  \iterm{Revenue} Innovating how the firm generate revenues, \eg by introducing new pricing models, such as freemium, or creating new demand in an uncontested market space --- a \term{blue ocean} \citep{kim2004}.
  \iterm{Enterprise} Innovating the role the firm plays is new or existing value chains, \eg by redefining the organizational boundaries or by specializing.
\end{enum}

There are several barrier to experimenting with a firm's business model \citep{chesbrough2010}, \eg managers might resist experiments which can threaten their value to the company \citep{amit2001}; organizations with established business models that are able to conceive the disruptive technology, but are not able to act on it \citep{bower1995,christensen1997,christensen2003}; and organizations can become path dependent if their established business model is successful \citep{chesbrough2010,tushman1986}.

According to \citet{chesbrough2010} one way of overcoming barriers to business model experimentation is to construct maps of business models, which provide a pro-active way to experiment with alternative models. This is essentially about understanding how the business model works for the organization and how it can be changed to improve the organization's chance of success.

% \todo{Dra inn discovery-driven fra McGrath i LRP}

The essential knowledge we can draw from this, is the importance of continually innovating the business model. Thus, exploration, not only exploitation, as discussed by \citet{march1991}, is vital to succeed.

% \subsection{Taxonomies, typologies \oldand the atomic business models}
% 
% \todo{På grunn av endringer i begynnelsen, henger ikke denne delen på greip lengre.}
% 
% In the literature the term business model has been used for various things \citep{osterwalder2005}, \eg parts of a business model, types of business models and concrete real world instances of business models.
% 
% Several academics have created taxonomies of business models \citep[\eg][]{timmers1998,weill2001,tapscott1998,tapscott2000,alt2001,rappa2010}, \ie creating possible categorizations of business models into a number of typologies based on various criteria \citep{pateli2004}. \citet{osterwalder2005} see the work on taxonomies as business models at a conceptual level. 
% 
% Using these taxonomies can \eg help in an ex post classification of a business model in order to conceptualize it, which can help an organization modify certain elements of their existing business model \citep{petrovic01}.
% 
% These taxonomies have been created based on different points of view. 
% 
% \todo{Scale models vs role models in \citet[\p{157}]{badenfuller2010models}.}
% 
% \citet{weill2001} introduced the term \term{atomic business model}. They identified eight such atomic business models, which could be implemented either by itself or combined with other atomic models. They saw atomic business models as a conceptual framework for understanding the complex reality of business models, and identified three ways to use them:
% 
% \begin{items}
%   \iterm{Atomic models as pure types} They describe the essence of a business model, and understanding them gives insight into successfully operating this type of model.
%   \iterm{Atomic models as building blocks} The atomic models can be combined --- into both viable and non-viable combinations.
%   \iterm{Decomposition of e-business initiatives} The atomic models allows for decomposition of initiatives to better understand the implementation requirements, \ie by decomposing to better understood models.
% \end{items}
% 
% As we will discuss later in this paper, freemium can be seen as an atomic business model.
% 
% Contrasted with \citet{weill2001}, some authors \citep[\eg][]{timmers1998,rappa2010} have classified real-life initiatives, such as Amazon or eBay, even though these can be seen as multiple atomic models combined. 
% 
% A danger in these taxonomies is looking at business models as a static concept. For an organization, a dynamic view on the business model it essential. \citet{linder2000} talk about change models that are the core logic for how a firm will change over time. \todo{én eller to setninger til om dette.}
% 
% There has been a great deal of research in this area, but going deeper into this research is outside the scope of this thesis.
% 
% The goal of this paper is to create a typological description of freemium.

\subsection{Framework --- The Business Model Canvas} 
\label{section:businessmodels:canvas}

To look deeper into business models the framework in \citet{osterwalder2010} --- \term{The Business Model Canvas} --- can be used. This framework is mentioned in \citet{chesbrough2010} as one example of a pro-active way for an organization to experiment with alternative models. The framework consist of nine \term{building blocks}, illustrated in which are defined as follows:

\begin{items}
  \iterm{Customer Segments} The different groups of people or organizations the company aims to reach and serve. Based on this decision, a business model can be carefully designed around a strong understanding of specific customer needs.
  \iterm{Value Propositions} The bundle of products and services that create value for a specific customer segment. The value proposition is the reason why customers choose one company over another.
  \iterm{Channels} How a company communicates with and reaches its customer segments to deliver a value proposition. Communication, distribution, and sales channels comprise a company's interface with customers. Finding the right mix of channels to satisfy how customers want to be reached is crucial in bringing a value proposition to market.
  \iterm{Customer Relationships} The types of relationships a company establishes with specific customer segments. Relationships can range from personal to automated, and may be driven by the following motivations: Customer acquisition, customer retention, and boosting sales, \ie upselling.
  \iterm{Revenue Streams} The cash a company generates from each customer segment. A revenue stream can either come from transaction revenues from one-time customer payments or from recurring revenues from ongoing payments.
  \iterm{Key Resources} The most important assets required to make a business model work. Different key resources are needed depending on the type of business model. Key resources can be physical, financial, intellectual, or human.
  \iterm{Key Activities} The most important things a company must do to make its business model work. Key activities differ depending on business model type. Can be categorized into production, problem solving, and platform/network, \ie what \citet{stabell1998} term \term{chains}, \term{shops} and \term{networks}, respectively.
  \iterm{Key Partnerships} The network of suppliers and partners that make the business model work. Companies create alliances to optimize their business models, reduce risk, or acquire resources. This can either be to optimize and achieve higher economies of scale, to reduce risk and uncertainty, or to acquire particular resources and activities.
  \iterm{Cost Structure} All costs incurred to operate a business model. Low cost structures are more important to some business models than to others, therefore it can be useful to distinguish between two broad classes of business model cost structures, cost-driven and value-driven, which is similar to \term{cost leadership} and \term{differentiation} discussed by \citet{porter1980}.
\end{items}

The connections between these building blocks can be seen in Figure \ref{figure:businessmodel:canvas}. As this is very comprehensive, all these building blocks will not be thoroughly discussed throughout the analysis and discussion. However, the interesting aspect of this with regards to this paper, is which of these are deemed interesting and important when examining cases.

\begin{figure}[h]
  \label{figure:businessmodel:canvas}
  \includegraphics[keepaspectratio=true,width=\textwidth]{canvas_real_with_names}
  \caption{The Business Model Canvas, illustrating the connections between building blocks.}
\end{figure}

\subsection{Business Model Theory \oldand Freemium}
\label{section:businessmodels:freemium}

Freemium has been described in several ways, \eg as a business model \citep{wilson2006}, revenue model \citep{parker2008}, pricing strategy \citep{lunn2010}, distribution model \citep{dahlquist2010}, marketing technique \citep{semeria2009}, and architectural model \citep{bhullar2010}. In this paper neither of these are corrected or falsified, as how freemium is categorized is of no importance to the essence of this paper. This paper is based on the premise that freemium will have effect on decisions regarding most or all the different elements in the business model. Lets look closer at some of these.

For freemium, an important question is who pays and who gets the product or service for free. Failing to segment the customers properly can end in trying to sell the product or service to those that are not willing to pay. A wrong segmentation can also be that those that are segmented as free users are actually not interested in ever converting to paid users, or otherwise paying for themselves. Thus, a good customer segmentation seems essential for freemium.

As freemium entails giving away the product or service for free, possibly into perpetuity, the chosen channels must have a very low cost, at least for the free users. An example of this is VoIP provider Skype, which is almost entirely based on peer-to-peer technology, lowering their marginal cost to very close to zero.

For free users, we can assume that automation is key when it comes to customer relationships. The less automated the customer relationships are, the more the company will lose on the free users. Thus, customer acquisition must be cheap, the customer retention rate must be high, and there must be enough conversion from free to premium. If either of these three are present, there appear to be problems with using freemium. 

Understandably the cost structure is one of the essential elements of a freemium based business model. If an organization is not able to have a very low cost of operation, it is not possible to use freemium. At least the cost structure must be sufficiently low for the the free users, and sufficiently low on average over all users to justify giving a product or service for free.

As we see, freemium will have a clear impact on a company's business model. Freemium helps in capturing, understanding, and visualizing the business logic of giving away something for free and charging a premium for something else.