\chapter{Conclusion}
\label{chapter:conclusion}

This paper has explored how and why companies have succeeded and failed with freemium. One of the most intriguing findings is the apparent success of some of the case companies, showing that it is possible to create a viable business model based on freemium. This success is, however, predicated on a set of conditions, seven of which are proposed in this paper.

All the companies examined in this paper focus on freemium as a marketing tool. It is, however, not a marketing tool every company can choose, as it affects both how the company creates, delivers, and captures value. As a marketing tool the most important benefit among the cases examined is the word-of-mouth effects the free version bring. However, it is important to ensure that this word-of-mouth attracts the right customers --- those that either can be converted to premium or in an other way bring more value than they cost to acquire and serve. 

As freemium entails possible giving a product or service away for free to the majority of customers into perpetuity, measuring and tracking is seen as an essential aspect as freemium is more opaque than many other models where each user pays from him- or herself directly. There have been a strong focus in the public sphere on using conversion rates as a measure of success of a freemium based model, but this paper show that the most important metric is the average lifetime value of a customer. 

For freemium the paying customers must pay for both themselves and for all the free users of the product or service. As potentially more than 90\% of the users are not paying, this demand very low marginal costs and a large addressable market. Otherwise it would respectively either be prohibitively expensive for the premium service, or not enough potential users to get the necessary amount of paying customers.

Seeing freemium through the lens of business models, information goods and free have given us a framework to research the model and get a better understanding of its inner workings. What we have found is that freemium is a potentially very powerful tool, but that it is essential to understand how it works and what it implies for a company before it is chosen. As such, the proposed propositions can be used as criteria for an ex ante analysis of the design of a freemium based business model.

\section{Limitations}

Some elements concerning the weaknesses of the study need to be noted. First of all this paper is based on an unobtrusive measure, and there is a challenge to obtain enough information about the chosen cases. Lacking data can lead to wrong conclusions from the cases. In addition, as public information is what determine whether or not a case is considered relevant for this paper, a range of companies that otherwise might be interesting with regards to freemium cannot be explored.

Another problem with public information is that it can be tainted for commercial reasons. While exploring cases, far fewer cases were found that talked about a company's struggle with freemium than with their success with it. This was especially problematic when searching for the unsuccessful cases, which were important for this paper.

As the chosen cases often will be selected based on the sharing of a select few in the companies, rationalization might be a problem, \eg an investor rationalizing why an investment was made in a freemium-based company. Related to this is the fact that much available information on freemium is not stated in the context of why and how a company has succeeded or failed with freemium. Thus, the discourse might be misunderstood as it is stated in another context, and can therefore lead to key categories not being identified.

The internal validity of this paper could be increased through triangulation and respondent validation \citep{webb2000}.

\section{Practical implications}

The primary practical implication of this study is a broader understanding of that freemium entails and implies. This paper has proposed a set of propositions that can be used by practitioners in the process of designing a viable business model based on freemium. Based on these propositions a company can get an explicit insight into freemium, which enable decisions based on more than just hunches and anecdotal evidence, as has been mentioned earlier in this paper as a prevailing problem today.

Basing this study in business model theory has proven viable and has given suggestions to a better understanding of what freemium entails, and thereby a better understanding of the term freemium itself.

\section{Theoretical implications \oldand Future work}

This study commence a deeper look into freemium itself, an area which is currently not well understood from a theoretical standpoint. As freemium is growing in use among practitioners, understanding it theoretically is important, especially as many companies are having considerable problems with the model. With this paper's exploratory study of a young field an enhanced picture of freemium is provided, and based on this there is a desire for more specific studies to better understand each component of freemium. 

Freemium is academically in its infancy, and it is therefore endless possibilities for future research. Based on the most interesting things discovered during the writing of this paper, the following avenues are seen as especially interesting.

\subsection{Critical Success Factors}

As this paper looks closer at what is \emph{sufficient} to succeed with freemium, research is needed on what is \emph{necessary}. The concept of Critical Success Factors (CSF) is defined by \citet{boynton1984} as \q{those few things that must go well to ensure success for [\ldots] an organization. [\ldots] CSFs include issues vital to an organization's current operating activities and to its future success.} Thus, we can see CSFs as what a firm must do well to succeed with the implementation \citep{weill2001} of freemium, and it is therefore seen as a highly interesting avenue of research.
   
\subsection{Behavioral Economics}

Behavioral economics explores the way in which our irrational behavior affects economies \citep{camerer2004}, and according to \citet[\p{275}]{simon1997} emphasizes \q{the factual complexities of our world.} Key themes of behavioral economics include heuristics, cognitive biases, bounded rationality and market inefficiencies \citep{simon1997}, all of which are interesting from a freemium perspective. Thus, there are several interesting avenues to look at freemium through the lens of behavioral economics, \eg where to put a boundary between free and premium, pricing, customer segmentation, and conversions from free to premium.
 
\subsection{Blue Ocean}

\citet{kim2004} argue that \term{Blue Oceans} is about breaking the value/cost trade-off, and that it is possible to pursue both differentiation \emph{and} low cost. Seeing this in relation to freemium is very interesting as freemium entail possible giving away a product or service for free into perpetuity for some users at the same time as there is fierce competition, thus pressuring towards both lower prices and differentiation. 

\subsection{Freemium Economics}

Many practitioners are discussing conversion rates \citep[\eg][]{anderson2008,chen2009,loan2009,asay2009}, however, a more systematic discussion on the economics of freemium is needed. This entail a closer look at the lifetime value of customers, conversion rates, customer retention rates, the cost to acquire a customer, and the cost to serve a customer. More insight into these, and other, rates and number can give a better discussion as to how a profitable freemium based company is created.

\subsection{Pricing \oldand Free versus Paid}

This paper got very few insights into the pricing of freemium based products and services. From the theory a much higher focus on this was expected. This is, however, still seen as a very interesting avenue of research. According to \citet{varian1997} the fundamental problem for information goods is to set the prices such that those users that are able and willing to pay high prices do so. There are several interesting aspects of pricing that should be considered, \eg extremeness aversion and how to set the boundary between the free and paid version of a product or service.

\subsection{From a strategic point of view}

As this discussion have been entirely focused on seeing freemium through the lens of business models, work should be done to understand freemium in relation to other aspects of an organization, such as its strategy. Blue Ocean, as already mentioned, is one such avenue. Other interesting avenues can be the \term{Balanced Scorecard} \citep{norton1992}; \term{Chains}, \term{Chops}, and \term{Networks} \citep{stabell1998}; and {Five Forces} \citep{porter1985}.

\subsection{Competition}

With regards to competition, both competing \emph{with} freemium and competing \emph{against} freemium are interesting, especially from a practical point of view. Interesting avenues of research can \eg be how --- or rather, if --- a startup can use freemium to disrupt an incumbent, how to defend against a competitor using freemium, and competing with freemium against another freemium based organization.

\subsection{Business-to-business}

Among the case companies, two focused on business-to-business. However, one of these failed considerable. As this paper propose that large markets are one of the criteria that must be present to succeed with freemium, and as business markets are inherently smaller than consumer markets, freemium need more research in the business-to-business context. An interesting aspect is how a freemium product or service is perceived by buyers as it can be used for free, at least for some organizations. An interesting questions is whether these organizations will perceive a free product as having less quality than a premium-only product or service.

\subsection{What happens if everyone uses freemium?}

The findings in this paper suggests that the most important aspect of freemium is marketing. But what happens if everyone uses freemium? In a world with limited attention, we can assume that word-of-mouth effects will decrease over time, thereby lowering the effect of freemium. We see that one of the examined cases, LogMeIn, already state that they use freemium to fight off new entrants, as they will not be able to disrupt LogMeIn on price. Thus, examining the effects of freemium and its potential future, is of significant practical value.

% \subsection{The importance of word-of-mouth}
% 
% \todo{!!!}
% 
% \subsection{Freemium, B2B}
% \subsection{Understanding the metrics}
% 
% Conversion rate, retention rate, cost of acquisition, cost to serve, lifetime value.