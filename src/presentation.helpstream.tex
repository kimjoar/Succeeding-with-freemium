\section{Case 4 --- Helpstream}

Helpstream\footnote{The presentation of Helpstream is based on their sharing of information in \citet{warfield2010,warfield2010b,warfield2010c}.} delivered a social customer service and relationship management (\abbr{CRM}) system, based on \abbr{SaaS}, which they explained as \q{community-driven customer service} --- customers helping each other resolve issues and answer questions. The company was founded in 2004, and went out of business in March 2010. The company had about \$8.6 million \abbr{USD} in funding.

The company was primarily \abbr{B2B} oriented,\footnote{They had consumer \emph{users} which increased the value for these businesses by being those who contributed to the customer service, but these could not upgrade to a premium version.} and at the end they had 240 users, of which 40 was on the premium service. There was, however, more than 500,000 registered users. Helpstream built an underlying technology that enabled them to achieve a cost of \$0.05 USD, 5 cents, per registered user per year, which Bob Warfield, the former \abbr{CEO} of Helpstream, states as \q{extremely low.} 

An interesting aspect of Helpstream, is that they had problems with the registered businesses who used freemium. Warfield states that \q{the biggest problem was that it was not attractive to the \emph{right kind} of customers,} as freemium was \q{self-selecting customers who were most likely to be willing to pay, well, nothing!} In addition, even though they had 500,000 users, these were not customer prospects for Helpstream, and all used the service for free. Thus, the company was not able to generate enough customers to survive.

Another of the company's problem, was that they had internal problems relating to the use of freemium, especially from the Sales and Marketing departments. The problem was Sales saw freemium as something that took away from their commission, and that they then became \q{actively hostile} --- Sales saw freemium as competition. Explaining the problem this created in the organization, Warfield states that \q{it became almost impossible to talk about the Freemium without polarizing armed camps, which eventually made it a sacred cow.}

The primary value Helpstream gained from using freemium was that \q{the free users gave us a tremendous amount of feedback and testing.} It also enabled the company to show momentum in terms of the number of customers. Despite this, Warfield recommend using a trial for \q{a company like Helpstream.} Looking back, he see two elements that could have helped the company; lowering the adoption friction and reducing the transaction friction --- thus, making it easier to use and easier to pay, especially as the software once registered \q{was sterile and did not give them an ability to see it in the context of their intended use.}

% \url{http://smoothspan.wordpress.com/2010/03/22/my-startup-track-record/}
% \url{http://sanjose.bizjournals.com/sanjose/stories/2008/02/25/daily64.html}
% \url{http://smoothspan.wordpress.com/2010/03/25/freemiums-for-saas/}:
% 
% \begin{items}
%   \item [The original CEO] wanted to price everywhere along the demand curve so as not to leave a flank exposed at some price point.
%   \item So, prospects would get a free workspace and play with it a bit, but it was sterile and did not give them an ability to see it in the context of their intended use. 
%   \item I looked at the usage statistics to determine who was deriving real value.  A hospital using the Freemium to deliver an IT Help Desk to hundreds of people was clearly deriving real value.  Based on these statistics, I set limits on the Freemium.
% \end{items}
% 
% \url{http://smoothspan.wordpress.com/2010/04/05/references-and-pricing-for-saas-startups/}:
% 
% \begin{items}
%   \item The first task when selling business software is getting anyone to pay anything for it and then be willing to talk about it so you have references.
%   \item Give it away in order to get a list of referencible customers started.  It’s very hard to sell business software without some references.
%   \item The initial references are going to come from networking.
%   \item This isn’t about extracting revenue.  This is about building credibility and learning something about the world’s reaction to you product. 
%   \item At Helpstream we started out at about half the cost of the low-cost players.  By the end of my tenure we were at half the cost of the highest cost players, and that by dint of discounts and much lower friction (more on lowering friction in a future post) in terms of professional services to install.
%   \item Most SaaS companies have way too complicated pricing.
%   \item If billing is a manual process, that’s when it becomes a hassle.
% \end{items}
% 
% \url{http://smoothspan.wordpress.com/2010/10/04/a-pair-of-killer-pricing-articles-for-startups/}:
% 
% \begin{items}
%   \item Think of Freemiums and Low Prices as Marketing.
%   \item The more value you recieve, the more you should pay.
%   \item It’s a beautiful thing to outflank the competition with a qualitatively different pricing model.  Qualitative means your pricing isn’t just cheaper, but it works differently.
%   \item Get a good sense of how long it should take them to evaluate your product, get all the ducks in a row on their side to make a purchase, and then use that figure, whatever it turns out to be.  Getting all the players in an org to agree on a Social CRM system in 14 days was never going to happen for Helpstream.
% \end{items}
% 
% Det med å gi det gratis til \q{feil} brukere, er det samme Sixteen Ventures skriver om Ning\footnote{\url{http://sixteenventures.com/blog/exploring-ning-post-freemium-pricing-page.html}}.