\section{Implications from Theory}
\label{section:implications}

During the preceding chapter relevant theory for exploring freemium have been examined --- business models, information goods and free. These are all elemental in understanding freemium, but is not enough to discuss the research question in this paper, namely how and why companies have succeeded and failed when using freemium. However, from these three areas of research we have a foundation to build a fruitful discussion aimed at answering this question.

There are some notable implications from the preceding theory. First of all, a business model have been defined as the rationale for creation, delivery and capturing of value. Thus, for an organization to profitably use freemium, it can be assumed that all these three elements will have a central position. Freemium has also been mapped to the \emph{Business Model Canvas,} which showed that freemium will have an impact on many disparate components of a business model. This thus imply that an organization must be well-aware of the choices they are making when designing a freemium based business model, as it is not only related to the revenue stream or some other component, but to a range of components.

From information goods we have the core tenets, value-based pricing and versioning, which are both assumed to be fundamental in understanding pricing of freemium-based products or services. From our last endeavor, free, we found that decreasing the price to zero increases perceived value considerable more. We also found three elements which is assumed to be important for freemium; network externalities, demonstration, and word-of-mouth.

As the remainder of this paper investigates empirically the research question, the notions that have been presented in this chapter are brought in to the analysis and discussion.