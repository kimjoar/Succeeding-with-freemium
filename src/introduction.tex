\chapter{Introduction}
\label{chapter:introduction}

The aim of this paper is to gain a better understanding of how a company can design a viable business
model around \term{freemium}. The word freemium is a combination of \q{free} and \q{premium}, and the concept was popularized by \citet{wilson2006} as:

\begin{quote}
Give your service away for free, possibly ad supported but maybe not, acquire a lot of customers very efficiently through word of mouth, referral networks, organic search marketing, etc, then offer premium priced value added services or an enhanced version of your service to your customer base.
\end{quote}

Thus, freemium is a concept whereby a company offers something for free, with no requirements to purchase now or later, in order to build a large user base, of which, at least, some purchase the premium priced offer. This is then a change from what economists call tying, as the customer is not conditioned to buy the premium version --- it is not indispensable.

The primary enabler of freemium is the decreasing cost of digitized products and services on the Internet \citep{anderson2009}. But even though the Internet has enabled freemium, many of the companies that have based their business model on it is now facing considerable difficulties, including several that have gone bankrupt  \citep{darlin2009}. In the fiercely competitive online market, it is enticing to give away a product for free in order to grow rapidly, but many companies have not been able to capture enough value to survive.

Several notable companies are using freemium today, including Spotify, the music streaming service offering unlimited streaming of music; the publicly traded LogMeIn, which provides remote access to computers over the Internet and is valued at more than \$1 billion \abbr{USD}; and the VoIP provider Skype, which allows users to make voice calls over the Internet --- for free to other Skype users --- and who is now filing for an initial public offering.
% was bought, and later sold, by eBay for \$2.5 billion \abbr{USD} in 2005, and 

The term \emph{freemium} has been recognized and adopted in academic circles \citep[\eg][]{doerr2010,shuen2008,teece2010,osterwalder2010}, and according to The New York Times freemium is becoming the most used business model among web-based startups \citep{miller2009}. But while it has been heavily used the last years, there has been little research specifically on the concept, and it is still uncertain as to whether or not business models based on freemium works \citep{teece2010}.

As freemium is becoming heavily used, though seemingly often misused, understanding it is important both from a practical and academic point of view. Following this the essence of this paper is exploring how and why companies have succeeded and failed when basing their business model on freemium.

% Following this the research question for the paper is:

% \q{What are the critical success factors for designing a viable business model around freemium?}

Having a viable business model --- that is, where the organization is capable of generating profit from the business model \citep{gordijn2001} --- is essential to succeed \citep{magretta2002}. The viability is tied both to the value the business model creates and to the way it capture value and generate revenue \citep{shafer2005}. An essential problem for Internet companies is that they have struggled to create viable business models \citep{teece2010}, something many companies experienced when the \q{dot-com bubble} burst.

% The concept of Critical Success Factors (CSF) was developed by \citet{daniel1968} and refined by \citet{rockart79}, and in this thesis CSFs are defined in line with \citet{boynton1984}, as \q{those few things that must go well to ensure success for [\ldots] an organization. [\ldots] CSFs include issues vital to an organization's current operating activities and to its future success.} Thus, we can see CSFs as what a firm must do well to succeed with the implementation \citep{weill2001} of freemium.

This paper has a keen focus on web-based products and services, and as such, this paper is limited to information goods. This is also the context in which \citet{wilson2006} defined freemium. Using freemium in other contexts might be possible, but answering this question is outside the scope of this paper.

% Achieving success with a business model relates to both the design and implementation of it, of which this paper primarily focus on the former. This paper aims at providing criteria for an ex ante analysis for choosing how to design a viable business model on freemium, in order to guide and inform a company with regards to critical decisions when using freemium.

Answering why and how freemium-based companies succeed or fail will be sought through a comparative case study. This will underline attributes of both companies that have succeeded and failed, and try to distinguish between these, to highlight what companies must do in order to achieve success when using freemium. An inevitable outcome of both the research question and how it was sought answered, is that this paper only answer what is sufficient, not necessary, to succeed with freemium.

% \todo{Blue Ocean prater \q{simultaneous pursuit of differentiation and low cost} noe som absolutt er freemium-relevant}

\section{The Structure of this Thesis}

This paper starts with a brief chapter on freemium, which aim to give a better understanding of what the concept entails, how it is used, and its position in the very competitive online market. Following this the relevant theoretical aspects of the paper will be dealt with. This entails a brief review of literature related to business models, information goods and giving away things for free, and aim to give an understanding the theoretical aspects of freemium. The methodology chapter explains how the research was conducted and discusses the credibility of the findings, before the presentation chapter and the ensuing discussion chapter where the chosen cases are presented and discussed in relation to the relevant theory. Finally the conclusion ties together the findings in a summary, and discusses the implications for industry and theory. Additionally the conclusion include several possible future directions for research on freemium.