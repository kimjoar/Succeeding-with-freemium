\begin{abstract}
  This paper explores how and why companies have succeeded and failed when basing their business model on \emph{freemium}. Even though freemium is increasing in use by web-based companies, there is still no research specifically on the concept. This paper aims at rectifying this by commencing a deeper examination of freemium.
  
  In this paper success and failure was explored through a comparative case study, examining four companies' use of freemium. From this, eight empirical findings were proposed. First and foremost, this paper found that it is possible to design viable business models based on freemium. Seven conditions to achieve this viability were proposed, based on a company's ability to create value, deliver value and capture value.
  
  This paper found that as a precondition to choose freemium, low marginal costs are required, at least on the free version. Otherwise the premium version would be too expensive for the paying customers. Freemium also depends on the company having a large addressable market, as it entails possibly giving away a company's product or service for free for some users into perpetuity. The key metric is then the customer lifetime value --- especially as freemium is more opaque that more traditional models where each user pays from him- or herself directly. In addition, based on this opaqueness, tracking and measuring the costs and revenues associated with the users are found to be vital in order to succeed with freemium.

  As the number of paid customers is vital, this paper found that the free version must convince a large enough subset of users to convert to paying customers by conveying the added value of the premium version. Thus, there must be a focus on converting \emph{enough} users from the free version to the premium version. If the company is not able to convey this value and convert users, they will not be able to capture enough value to survive with freemium.
  
  Strong word-of-mouth effects were found to be vital to achieve success with freemium. This finding seemed to go to the core of why the case companies chose to use freemium. In all the four case companies examined freemium was primarily chosen based on the possible marketing effects of giving away the product or service to as many as possible, and especially as it could, and indeed did in the three successful cases, decrease the customer acquisition cost. However, the resulting rapidly increasing amount of customers for the case companies revealed the importance of automation of manually demanding activities, as the companies would otherwise not be able to keep the cost of their free users low enough.
\end{abstract}