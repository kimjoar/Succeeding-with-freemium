\chapter{Context}
\label{chapter:context}

To understand freemium, we must first look at it in relation to the Internet, how it is used, and its position in the very competitive online market. As there have been little research on freemium thus far, I introduce the term based on non-academic work.

\section{Free \oldand the Internet}

The concept of giving something away in order to entice customers is not new. Similar terms include shareware and freeware, but these are defined specifically for software \citep[\eg][]{liao2002}. Freemium on the other hand is about services. To exemplify this distinction, we can look at the music streaming service Spotify. It is not the downloadable software that matters, but the music --- the service. Other than that, the rationale is similar --- enticing customers to buy the premium version.

According to the Copenhagen Institute for Futures Studies \q{we are witnessing a pronounced flourishing of free content and services on the internet} \citep[\p{14}]{anarconomy2009}. They argue that this implies all things digital and mass produced will become free, while unique products and experiences will be worth more. Thus, all companies that have digitized products will be affected. One example of this is how Wikipedia disrupted the historically lucrative encyclopedia business. Bill Gates, founder and former CEO of Microsoft, have said that the Internet helps achieve a \q{friction-free capitalism} \citep{gates1995,gates2006}, as buyers and sellers are put in direct contact with each other, and where the cost of manufacturing and distribution approaches zero.

At the same time as the Internet is changing the economy by pressuring the price downwards \citep{anderson2009}, it enables new possibilities, \eg such as making it easier to price discriminate between customer segments, easier to reach a far bigger market at a cheaper price than for physical products, and providing opportunities for making money in entirely new ways, one of which is freemium. \citet{anderson2009} claims that \q{people are making lots of money and charging nothing.}

\section{What Freemium Is And Is Not}

As the definition of freemium is equivocal as to specifically what it is and is not, and therefore who are using freemium and who are not, a clarification is needed. For this paper, a company is said to be using freemium if at least one of their products is based on giving away a free version for some users into perpetuity, while others must pay for a premium version. In this paper premium is defined as advanced features, functionality, or related products and services \citep[in line with][]{pujol2010,iglesia2008,hayes2008}. From this we see that freemium can be both related to one specific product or to a company's suite of products. Later in this paper, examples of both these categories will be presented.

Defining specifically when a company is using freemium and when it is not, is inherently difficult because of the potential complexities which can be introduced when a product or service is digitized. However, it is outside the scope of this project to look closer at classifying specifically when a company is using freemium and when it is not.

\section{Free Versus Paid}
\label{section:context:freepaid}

According to \citet{rekhi2010}, a San Francisco based entrepreneur and former Entrepreneur In Residence at early stage VC Trinity Ventures, choosing where to divide the free and paid plans is the essential critical question when using freemium. 

% Legg til en lignende modell som på \url{http://nicolaspujol.com/two-sided-markets/business-models/freemium/}

Several practitioners have proposed typologies for freemium to better analyze this chasm. \citet{anderson2009} suggests five ways to differentiate free and paid: The free version can have less functionality than the paid version; less capacity, \eg the number of megabytes of pictures; be limited to a number of people; be free for some, \eg for non-commercial use, while others must pay, \eg for commercial use --- \ie differentiating by customer class; and lastly it can be a limited amount of time on the full featured product, often called a trial. 

Another suggestion is differentiating on quantity, features, or distribution \citep{pujol2010}. Differentiating by quantity can \eg be limiting on time, as also mentioned by \citet{anderson2009}; by maturity, \eg giving a subset of the user base the program for free while it is still under development and not ready for general consumption; or by letting paying customers get a time advantage over free users by letting them buy the time-sensitive information before it is provided free of charge at a later time. What \citet{pujol2010} terms distribution is similar to what \citet{anderson2009} terms customer class. Thus, we see that these typologies are similar, but \citet{pujol2010} has a far more inclusive quantity category, and therefore a less specific differentiation that the one presented by \citet{anderson2009}.

Both typologies mentioned see trials as a form of freemium, many practitioners, however, don't, and discusses it as a distinct concept different from freemium \citep[\eg][]{robles2009,hudson2009,fry2010}. Including trials lead to products such as Microsoft Office 10 and Adobe Photoshop CS 5 using freemium, something many disapprove. In this paper trials are not seen as freemium, and will therefore not be discussed further.

Not mentioned in the typologies we have examined, is what \citet{murphy2010} terms \term{alternative product}. He define this as a \q{very simple, single-purpose [product or service] that solves an immediate, 
specific and highly targeted need, but which promotes significant network effect and ecosystem 
opportunities well beyond that application} \citep[\p{14}]{murphy2010}. The primary differentiator to what the typologies discuss, is that this product is stand-alone and only an addition to the \q{flagship product}, which is not using freemium.

\newthought{To distinguish between} who should pay and who can use a product or service for free, \citet{chen2009} suggests that \q{the key is to create the right mix of features to segment out the people who are willing to pay, but without alienating the users who make up your free audience.} And this is exactly what \citet{hudson2009} see as a problem, stating that \q{It is very difficult to properly segment users and features such that you provide enough value to both paid and free audiences.}

% \section{Rates \oldand numbers}
% 
% \url{http://www.web2expo.com/webexny2010/public/schedule/detail/15128}:
% 
% Often, Freemium start-ups not only don’t know how they’re performing but don’t even know how to measure how they’re performing. Lifetime value is the lifeblood of this business model, so failing to understand whether they are moving the company toward success can present significant challenges.
% 
% cost of acquisition, cost to serve, conversion rates and retention rates.

\section{So, What Do Users Pay?}
\label{section:context:what}

In their report on freemium, the consultancy Sixteen Ventures claim that when using freemium \q{the majority of users likely have no intention of ever becoming a customer} \citep[\p{6}]{murphy2010}. In a regular non-free purchase of a product there is a \term{quid pro quo} --- money in exchange for the product --- which is not as clearly eminent in freemium, as most customers might \emph{never} pay for using the product or service. Is there then a quid pro quo for freemium, and if so, what is it?

According to Sixteen Ventures themselves the quid pro quo is about getting the users attention and utilizing this attention to make money --- \q{it only makes sense to seek ways to benefit from and monetize their usage of the system} \citep[\p{6}]{murphy2010}. With a somewhat different view, \citet{pujol2010} terms the quid pro quo as \term{mind share,} and defines it as \q{the development of awareness for the provider's brand and the consideration for purchase of future commercial products and service} \citep[\pp{2}{3}]{pujol2010}. In the former view the focus is on utilizing what the company can learn from users, while the latter focus on marketing aspects of freemium. Without taking any stands between these two positions, we see that there is still a quid pro quo, it is, however, different from the classical money in exchange for product.

\section{Researching Freemium}

Currently a third of the top grossing iPhone apps use a freemium model \citep{kim2010}. Freemium is also heavily used in Facebook applications, with the most famous actor, Zynga, now valued at \$5.5 billion \abbr{USD} \citep{morris2010}. In addition to the examples we have seen earlier, we see that freemium is used by several prominent actors. However, at the same time there are considerable uncertainties as many companies are experiencing considerable problems. \citet{rekhi2010} claims:

\begin{quote}
I believe we are still early in our understanding of [freemium] and to date most of the available analysis has been limited to anecdotal evidence, one-off case studies, tips \& tricks, and a few early overviews of what's been tried.
\end{quote}

Tongue-in-cheek we can say that generally the yiddish proverb \q{For example is not proof} seems appropriate, and clearly research on freemium is highly needed.

% There are several notable companies that are using freemium, \eg
% 
% \begin{items}
%   \iterm{Flickr} Image hosting and video hosting website. Free account users are allowed to upload 100 MB of images a month and 2 videos. In September 2010, Flickr reported that they host more than 5 billion images\footnote{\url{http://blog.flickr.net/en/2010/09/19/5000000000/}}. Was acquired by Yahoo! in March 2005. 
%   \iterm{Skype} Low-cost or free VoIP and video communication provider. Currently has more than 520 million registered users, of which X million are premium customers. Was acquired by eBay for 2.6 billion \abbr{USD} in 2005, and later acquired by a group of private investors. For the first six months of 2010, Skype reported revenue of \$406 million, and net income of \$13.2 million. Skype filed for an IPO in august 2010\footnote{\url{http://about.skype.com/press/2010/08/ipo.html}}. Skype says it has 560 million registered users, and 124 million who use it at least once a month. Of those, 8.1 million are paying an average of \$96 a year\footnote{\url{http://mediamemo.allthingsd.com/20100809/big-tech-ipo-of-the-day-skype-tries-to-dial-up-100-million/}}. \q{We can add new users and provide them with a wide range of communications tools at minimal incremental cost to us, allowing us to offer many of our products for free}\footnote{\url{http://sec.gov/Archives/edgar/data/1498209/000119312510182561/ds1.htm}}. \q{As our users continue to use Skype more often, they begin to recognize the full benefits of using our products and many migrate to using Skype as their preferred communications tool across a variety of connected devices.} \q{[the 'peer-to-peer´ architecture] provides us with a significant cost advantage}
%   \iterm{LogMeIn} Suite of software services that provides remote access to computers over the Internet, of which two of the applications in the suite is free of charge. Currently has 10.4 million registered users, of which 490 000 are premium users. Is publicly traded and has a market capitalization of 1 billion \abbr{USD}. LogMeIn is one of the case companies that will be inspected closer later in this paper.
%   \iterm{Spotify} Music streaming service offering unlimited streaming of selected music from a range of major and independent record labels.
%   \iterm{Craigslist} Centralized network of online communities, featuring free online classified advertisements. Serves more than 20 billion page views per month\footnote{\url{http://www.craigslist.org/about/factsheet}}. In the US Craigslist has been a central actor in disrupting the newspapers' classified ads\footnote{\url{http://pewinternet.org/~/media/Files/Reports/2009/PIP\%20-\%20Online\%20Classifieds.pdf}}. Craigslist are privately held, and don't disclose revenue and other economic numbers, but are estimated to bring in more than \$ 100 million with just 30 employees\footnote{\url{http://www.nytimes.com/2009/06/10/technology/internet/10craig.html}}.
% \end{items}
% 
% 
% Further, Rekhi claims that where to divide the free and paid plans is the essential critical question when using freemium. As for most web based pricing plans, it is technically easy to change the pricing or other aspect of the different pricing tiers when using freemium, but a difficulty is the possible user backlash that can result from this. According to \citet{anderson2009} there are at least five ways freemium can be divided:
% 
% \begin{enum}
%   \iterm{Features} Limited functionality, thus the product needs more features than in this class for the premium version.
%   \iterm{Capacity} Limited capacity, \eg the number of megabytes of pictures.
%   \iterm{Seat} Limited to a number of people.
%   \iterm{Customer Class} Free for some, \eg normal users, while others must pay, \eg businesses.
%   \iterm{Time} Limited amount of time on the full featured product, often called a trial.
% \end{enum}
% 
% However, there many don't see a time limited trial as freemium \citep[][\eg]{robles2009,fry2010}.
% \todo{Denne oppgaven tar ikke for seg de som kun har trial. Den ser altså kun på de som har en evig gratisversjon for noen av brukerne.}
% 
% 
% Charles Hudson, the man behind the freemium conference \emph{The Freemium Summit}, claims that \q{As a basic rule, I think freemium can work really well for products in which the free and paying users can peacefully coexist or where having more people engaged using the service actually makes it more useful for paying and free users.} On the other hand he also states that \q{It is very difficult to properly segment users and features such that you provide enough value to both paid and free audiences.} Phil Wainewright said is similarly, \q{The trick is to strike the right mix.}\footnote{\url{http://www.zdnet.com/blog/saas/how-to-make-freemium-pay/706}}
% 
% Om Malik, the founder of GigaOM Network and a former senior writer at Forbes, claims that there are \q{10 Commandments of a Successful Freemium App}\footnote{\url{http://gigaom.com/2009/09/01/how-freemium-can-work-for-your-startup/}. Retrieved Dec 7, 2010.}, which, amongst others, include:
% 
% \begin{items}
%   \item Focus deeply on one single domain;
%   \item Clearly define what is free and what is paid;
%   \item Build a subscription service into your application;
%   \item Encourage your customers to use your application often, for the more they use the application, more likely they are to establish a relationship with your company and that means you can sell them something new (or an upgrade) in the future; and
%   \item Data is the ultimate lock in. The more data that is stored inside the application, more difficult it is for customer to switch, because of the extra effort involved.
% \end{items}
% 
% Nicolas Pujol, who currently writes a book where freemium is a central compoenent, uses the term \term{mind share} to describe the free users currency. He defines mindshare as \q{the development of awareness for the provider’s brand and the consideration for purchase of future commercial products and services} \citep[\pp{2}{3}]{pujol2010}.
% 
% Simplistically, to become profitable when using freemium, the lifetime value of a company's paying customers needs to be greater than the cost it took to acquire them, plus, the cost of servicing all users (free or paying).
% 
% For freemium an often discussed element is the conversion rate. Taylor Buley of Forbes define a conversion rate as follows:
% 
% \begin{quote}
% [You give] away some services in hopes of getting users interested in ones that cost money. This upsell is called a conversion and most freemium businesses pay attention to the number of conversions relative to overall users, or at least active users, called a \term{conversion rate}.
% \end{quote}
% 
% Thor Muller, CTO of Get Satisfaction, has stated that \q{Freemium only works if the customer is delighted.}\footnote{At a Freemium Summit presentation.}
% 
% \url{http://www.inc.com/guides/2010/11/how-to-make-freemium-work-for-you.html}:
% 
% Box.net started using freemium in 2006, and saw sales spike 1,822\% in three years\footnote{Presentation named \q{6 Reasons You'd Be Crazy Not To Give Your Software Out For Free} at Web 2.0 Expo (\url{http://www.web2expo.com/webexny2010/})}. Before turning to freemium, Box.net had been a paid-only software service providing file storage online.
% 
% The biggest problem companies run into is that they stake too much in freemium without coordinating all the parts of their business.
% 
% \url{http://www.inc.com/magazine/20101101/go-ahead-raise-your-businesss-prices.html}:
% 
% Other notable CEOs, such as Jason Fried of 37Signals, has said that (By taking on a massive amount of users, he writes, his company might not be able to sustain its level of quality and success.)
% 
% Martin Kleppmann, co-founder of Rapportive, which adds rich contact profiles to your email, has created a set of 10 questions to consider before choosing freemium\footnote{\url{http://blog.rapportive.com/is-freemium-right-for-you}}:
% 
% \begin{enum}
%   \item How do users feel about your product, as a function of time? Products which increase in value over time are good candidates for freemium.
%   \item How good is your long-term retention? With good retention, you have a better chance of converting free users to paid users.
%   \item Does your product require behaviour change, or can people start using it gradually? If people can start using your product gradually, freemium might work.
%   \item Does your product have distinct modes of use for different audiences? If it does, you can be more creative with your freemium model.
%   \item What is your market like? If you are targeting a large unmet need, you should make your product free.
%   \item Are you targeting a premium niche? Free users are the opposite of premium.
%   \item Which metric do you use to separate free from paying users? Freemium makes sense if there's an obvious point to start charging.
%   \item Do you depend on word-of-mouth marketing? More users (even if they are free) = more mouths.
%   \item What kind of company do you want to build? You need lots of users if you want to take over the world.
%   \item What are your costs per user? They had better be low if you have lots of free users.
% \end{enum}
% 
% Simplify it down to one question: Do you have enough traffic/users where a 2\% to 3\% conversion rate will sustain your business?
% 
% \todo{An often overlooked (and super important) effect of offering a free plan is that it will decrease the number of paid users signing up to your service initially. In other words: you will make less money upfront. If the free plan is available that's what most people will go for. (Koble dette mot Shampanier-artikkelen)}
% 
% In what way are nonpaying customers valuable in a freemium pricing model? They keep your user from using a competing service (whether pay or free).
% 
% Diskutere two-sided markets og si at det ikke nødvendigvis er slik at noen av brukerene som må betale. Se f.eks. på Twitter, der Google og Microsoft betaler. 
% 
% 1. helps spread the word (lower customer acquistion costs) 2. \q{free} users are more likely to give you feedback, which will make your app better over time 3. you and your team feel good that you're able to share with the world something for free (just think of how many free apps help you during the day) 4. you can convert free users to paying customers. (in our case the app actually helps the users grow their business) 5. people seem to be willing to try out free apps vs trial apps more often
% 
% Paid user churn should be minimum: While the conversion to paid is important, it is equally important to ensure that a paid user continues paying across multiple cycles. Clearly, the more the churn, the lower the lifetime value of the user is going to be.
% 
% \q{Another powerful effect of using the free strategy is that it usually results in a far larger customer base using the free products, who become proponents for your company. This expanded footprint or market share can have a huge effect on the price that acquirers or investors are willing to pay for your company, as they recognize that even though these customers have yet to be monetized, they represent a great potential for future monetization.}\footnote{\url{http://www.forentrepreneurs.com/business-models/power-of-free/}}
% 
% Chris Anderson: \q{It's an inversion of the old free sample promotion: Rather than giving away one brownie to sell 99 others, you give away 99 virtual penguins to sell one virtual igloo.}\footnote{\url{http://online.wsj.com/article/SB123335678420235003.html}} og \q{With physical stuff, samples must be doled out sparingly -- there are real costs to be paid. With bits, the free versions are too cheap to meter and can be spread far and wide.}
% 
% Mark Evans claims that \q{Freemium is flawed because most people don’t need more features than what they can use for free.}\footnote{\url{http://www.markevanstech.com/2008/10/17/freemium-is-not-a-business-model/}} In the same category is Steve Hodson, claiming that \q{As idealistic as the freemium model may appear it may not be the best answer for any of the parties involved.}\footnote{\url{http://mashable.com/2008/06/13/freemium/}}
% 
% Jason Fried: \q{I would be careful not to give away too much. If you have something free that is really very close to the paid thing, people aren’t going to pay you for it.}\footnote{\url{http://network.businessofsoftware.org/video/video/show?id=2352433:Video:2016}}
% 
% David Lenehan, co-founder of PollDaddy: \q{In my head freemium breaks down to consumers (free) and businesses (pay).}\footnote{\url{http://www.markevanstech.com/2008/10/23/another-stab-at-the-freemium-thesis/}}
% 
% \url{http://www.web2expo.com/webexny2010/public/schedule/detail/15128}:
% 
% Often, Freemium start-ups not only don’t know how they’re performing but don’t even know how to measure how they’re performing. Lifetime value is the lifeblood of this business model, so failing to understand whether they are moving the company toward success can present significant challenges.
% 
% cost of acquisition, cost to serve, conversion rates and retention rates.
% 
% \todo{Nevne SaaS?}
% 
% \q{Reach traditionally impenetrable markets: your product will seep into new – and sometimes surprising – ecosystems} --- Aaron Levie, founder of Box.net (\url{http://www.web2expo.com/webexny2010/public/schedule/detail/15046})
% 
% \q{the majority of people who are on pay started on pay,} Jason Fried\footnote{\url{http://mixergy.com/bootstrapping-37signals/}}