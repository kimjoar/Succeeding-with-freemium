\section{Case 3 --- MailChimp}

MailChimp\footnote{The presentation of MailChimp is based on their sharing of information in \citet{chestnut2010,chestnut2010b,chestnut2010c}.} delivers an email marketing service to design, send, and track HTML email campaigns. The company launched in 2001, and was a premium-only service for eight years, until they changed to freemium in September, 2009. At that point they had 85,000 users, which grew by five times to 450,000 users in the following 12 months. They are now adding 30,000 new users each month, of which more than 13\% are paying customers.

In the 12 months from September 2009 when they chose to go freemium, MailChimp increased their number of paying customers with 150\% and their profit with more than 650\%. According to the company the primary reason for their increased profit, is that their cost of customer acquisition had dropped considerable. In the last quarter alone the cost fell by 8\%. 

From April 2010 through August 2010 the company saw an increase on their largest pricing plan from 12\% to 20\% of paying customers. These 20\% account for 65\% of MailChimp's total revenue, up from 48\% in April. Thus, a very important growth factor during the 12 months of freemium was on their most expensive plan --- a growth that was considerably cheaper to achieve with freemium.

Ben Chestnut, the \abbr{CEO} of MailChimp, explains that there were two reason for changing to a freemium model. First of all they had an affordable self-serve product which was scalable and had good abuse prevention, and secondly cloud computing made \q{all [this] even cheaper.} Together this means that they had an highly automated system that scaled well, and that was becoming cheaper and cheaper for them to operate and manage. Based on this, they saw an opportunity to grow by using freemium.

However, abuse was one of the elements that troubled the company after going freemium. They saw an increase of more than 350\% in monthly abuse related issues. Ben Chestnut explained that solving this problem manually would \q{would at least have taken 30 human beings,} which is almost as many as the 38 that are currently employed at MailChimp. This is one of the problems they have had to tackle by automation to be able to stay with freemium.

An interesting aspect of MailChimp's change to freemium, is, as Ben Chestnut states, that \q{it was also a way to thank our customers.} They have seen many users change to a free model, and they are now doubling the free plan and are expecting to see even more convert down to free --- but at the same time, see a considerable increase of their premium plans.

% \url{http://www.mailchimp.com/blog/going-freemium-one-year-later/}:
% 
% \begin{items}
%   \item We're delivering roughly 700 million emails per month
%   \item Here’s another piece of history. Ever since inception, I’ve been fascinated with the art and science of pricing. I’ve tinkered with pay-as-you-go and monthly plans for \$9, \$9.99, \$25, \$49, \$99.99 and so on. We’ve changed our pricing models at least a half-dozen times throughout the years, and along the way we tracked profitability, changes in order volume, how many people downgraded when we reduced prices, how many refunds were given, etc. We’re sitting on tons of pricing data. When we launched our freemium plan in 2009, you betcha we used that data to see what would happen if we cannibalized our \$15 plan.
%   \item Build up that \q{1} before you chase the \q{10.} After you've got your \q{1} all set, use VCs to help you chase after that \q{10} (if you must). That's my personal opinion.
%   \item I don’t want anybody to think we launched freemium without a lot of careful thought and planning. And ever since launching, we’ve monitored our stats, and learned more about our freemium users, their impact on our business, and ultimately, used that data to decide whether or not we should double it this year.
% \end{items}
% 
% With 20\% of their user base (450k) on 10,000+ subscribers per month (\$150 pcm) I make that \$13,500,000 per month. And with that being 65\% of their total revenue, they are making a cool \$20m a mth. Not bad, not bad at all.
% 
% So, at least in my case, Freemium works best for products that either directly or indirectly contribute to making money.
% 
% I'd argue that freemium works best when the \q{premium} part has value for the user. But, making money (i.e. directly contributing to revenue/profit) is just one way to have value. Saving time is another.
% 
% I'd also rather the user tries out my app rather than the competition's
% 
% \url{http://www.youtube.com/watch?v=29j0kyCi0dM}:
% 
% \begin{items}
%   \item Around for 10 years
%   \item \q{Because of the cloud [...] we noticed that everything was getting cheaper.}
%   \item More than 200\% revenue growth
%   \item +354\% abuse related issues per month.
%   \item \q{You are going to have abuse related events if you are successful}
%   \item 38 person company
%   \item Automated handling of abuse related events (Koble mot Osterwalder)
%   \item \q{It would at least have taken 30 human beings to do this}
% \end{items}